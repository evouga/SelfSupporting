%%% The ``\documentclass'' command has one parameter, based on the kind of
%%% document you are preparing.
%%%
%%% [annual] - Technical paper accepted for presentation at the ACM SIGGRAPH 
%%%   or SIGGRAPH Asia annual conference.
%%% [sponsored] - Short or full-length technical paper accepted for 
%%%   presentation at an event sponsored by ACM SIGGRAPH
%%%   (but not the annual conference Technical Papers program).
%%% [abstract] - A one-page abstract of your accepted content
%%%   (Technical Sketches, Posters, Emerging Technologies, etc.). 
%%%   Content greater than one page in length should use the "[sponsored]"
%%%   parameter.
%%% [preprint] - A preprint version of your final content.
%%% [review] - A technical paper submitted for review. Includes line
%%%   numbers and anonymization of author and affiliation information.

\documentclass[annual]{acmsiggraph}

\usepackage{par06}
\usepackage{amsmath}
\usepackage{amssymb}
\usepackage{amsthm}

\def\<{\mathchoice{\big\langle}{\langle}{\langle}{\langle}}
\def\>{\mathchoice{\big\rangle}{\rangle}{\rangle}{\rangle}}
\def\II{\mbox{I\hskip-0.1exI}}
\def\pu{{\partial\over\partial u}}
\def\du{{d\over d u}}
\def\hinauf#1#2{{\mathop{#1}\limits^{\vbox to-.32564ex{\kern-.4652ex
   \hbox{\normalfont #2}\vss}}}}
\def\dddot#1{\hinauf{#1}{...}}
\def\ddddot#1{\hinauf{#1}{....}}
\def\str{/\penalty10000\hskip0pt}
\def\d#1dot#2{\dddddot{#2}}
\newtheorem{theorem}{Theorem}
\newcommand{\todo}[1]{\textcolor{red}{#1}}
\newcommand{\secref}[1]{(\S\ref{#1})}
\def\be{\begin{equation}}
\def\ee{\end{equation}}

%%% If you are submitting your paper to one of our annual conferences - the 
%%% ACM SIGGRAPH conference held in North America, or the SIGGRAPH Asia 
%%% conference held in Southeast Asia - there are several commands you should 
%%% consider using in the preparation of your document.

%%% 1. ``\TOGonlineID''
%%% When you submit your paper for review, please use the ``\TOGonlineID''
%%% command to include the online ID value assigned to your paper by the
%%% submission management system. Replace '45678' with the value you were
%%% assigned.

\TOGonlineid{0043}

%%% 2. ``\TOGvolume'' and ``\TOGnumber''
%%% If you are preparing a preprint of your accepted paper, and your paper
%%% will be published in an issue of the ACM ``Transactions on Graphics''
%%% journal, replace the ``0'' values in the commands below with the correct
%%% volume and number values for that issue - you'll get them before your
%%% final paper is due.

\TOGvolume{0}
\TOGnumber{0}

%%% 3. ``TOGarticleDOI''
%%% The ``TOGarticleDOI'' command accepts the DOI information provided to you
%%% during production, and which makes up the URLs which identifies the ACM
%%% article page and direct PDF link in the ACM Digital Library.
%%% Replace ``1111111.2222222'' with the values you are given.

\TOGarticleDOI{1111111.2222222}

%%% 4. ``\TOGprojectURL'', ``\TOGvideoURL'', ``\TOGdataURL'', ``\TOGcodeURL''
%%% If you would like to include links to personal repositories for auxiliary
%%% material related your research contribution, you may use one or more of
%%% these commands to define an appropriate URL. The ``\TOGlinkslist'' command
%%% found just before the first section of your document will add hyperlinked
%%% icons to your document, in addition to hyperlinked icons which point to
%%% the ACM Digital Library article page and the ACM Digital Library-held PDF.

\TOGprojectURL{}
\TOGvideoURL{}
\TOGdataURL{}
\TOGcodeURL{}

%%% Replace ``PAPER TEMPLATE TITLE'' with the title of your paper or abstract.

\title{Design of Self-supporting Surfaces}

%%% The ``\author{}'' command takes the names and affiliations of each of the
%%% authors of your paper or abstract. The ``\thanks{}'' command takes the
%%% contact information for each author.
%%% For multiple authors, separate each author's information by the ``\and''
%%% command.

%\author{Roy G. Biv\thanks{e-mail: roy.g.biv@aol.com}\\ Starbucks Research %
%\and Ed Grimley\thanks{e-mail:ed.grimley@aol.com}\\Nigel Mansell\thanks{nigelf1@msn.com}\\ Grimley Widgets, Inc. %
%\and Martha Stewart\thanks{e-mail:martha.stewart@marthastewart.com}\\ Martha Stewart Enterprises \\ Microsoft Research}

%%% The ``pdfauthor'' command accepts the authors of the work,
%%% comma-delimited, and adds this information to the PDF metadata.

\pdfauthor{Anonymous}

%%% Keywords that describe your work. The ``\keywordlist'' command will print
%%% them out.

\keywords{TODO}

%%% The ``\begin{document}'' command is the start of the document.

%%% If you have user-defined macros, you may include them here.

% example of a user-defined macro called ``remark.''
% \newcommand{\remark}[1]{\textcolor{red}{#1}}

\begin{document}

%%% A ``teaser'' image appears under the title and affiliation information,
%%% horizontally centered, and above the two columns of text. This is OPTIONAL.
%%% If you choose to have a ``teaser'' image, it needs to be placed between
%%% ``\begin{document}'' and ``\maketitle.''

%\teaser{
%   \includegraphics[height=1.5in]{images/sampleteaser}
%   \caption{Spring Training 2009, Peoria, AZ.}
%}

%%% The ``\maketitle'' command must appear after ``\begin{document}'' and,
%%% if you have one, after the definition of your ``teaser'' image, and
%%% before the first ``\section'' command.

\maketitle

%%% Your paper's abstract goes in its own section.

\begin{abstract}

\todo{TODO}

\end{abstract}

%%% ACM Computing Review (CR) categories.
%%% See <http://www.acm.org/class/1998/> for details.
%%% The ``\CRcat'' command takes four arguments.

\begin{CRcatlist}
  \CRcat{I.3.5}{Computer Graphics}{Computational Geometry and Object Modeling}{Curve, surface, solid, and object representations};
\end{CRcatlist}

%%% The ``\keywordlist'' command prints out the keywords.

\keywordlist

%%% The ``\TOGlinkslist'' command will insert hyperlinked icon(s) to your
%%% paper. This includes, at a minimum, hyperlinked icons to the ACM article
%%% page and the ACM Digital Library-held PDF. If you added URLs to
%%% ``\TOGprojectURL'' or the other, similar commands, they will be added to
%%% the list of icons.
%%% Note: this functionality only works for annual-conference papers.

\TOGlinkslist

%%% The ``\copyrightspace'' command 
%%% Do not remove this command.

\copyrightspace

%%% This is the first section of the body of your paper.

\section{Introduction}

\todo{TODO gentle introduction to self-supporting surfaces, assumptions about the model, importance to architecture etc}

\subsection{Self-supporting Surfaces}
Consider a surface in $\mathbb{R}^3$ given by a height field $z(x,y): \Omega \to \mathbb{R}$ over a region $\Omega$ of the plane, and suppose loads $F(x,y): \Omega \to \mathbb{R}$ act vertically on the surface. From the calculus of variations it can be shown \cite{TODO} that the surface is self-supporting if and only if there exists a field of $2\times 2$ matrices $M(x,y)$ over $\Omega$ satisfying
\begin{align}
\nabla^T \cdot (M\nabla z) &= F  \label{eq:conds}\\
\nabla^T \cdot M &= 0 \notag \\
M &\geq 0, \notag
\end{align}
where $\nabla^T \cdot A$ is understood as the divergence operator applied to the columns of a matrix $A$.

The condition $\nabla \cdot M = 0$ is exactly the integrability condition on $\det(M) M^{-1}$, so that locally (and globally, if $\Omega$ is simply-connected) 
\begin{equation*}
M = \det(\nabla^2 \phi) (\nabla^2 \phi)^{-1}
\end{equation*}
for a convex Airy stress potential $\phi:\Omega \to \mathbb{R}$.

The surface's position might be fixed at the boundary (such as where it is anchored to the ground), $z(x,y) = z_0(x,y)$. There are no constraints on the stresses at these points. Alternatively, the surface can be free at the boundary point, in which case the stress must vanish along that boundary, $\nabla z^T M \hat{\nw} = 0$.

\subsection{Related Work}
\todo{TODO Related work}

% architecture (Block and precursors), mechanics (Fraternali, Braun), graphics (Killian, Whiting), isotropic geometry, relative curvature}
block~\cite{Block06},\cite{Block07}, \cite{O'Dwyer98}, \cite{Heyman95}, \cite{Livesley92}, etc

graphics~\cite{Whiting09}, \cite{Kilian2005}, etc

stress surface~\cite{Fraternali2010}, \cite{Ash1988}, \cite{Giaquinta1985}, etc

isotropic geometry~Strubecker, \cite{Koenderink2002}, etc

PQ meshes~\cite{Schiftner2010}, \cite{Glymph2004}, \cite{Pottmann2007b}, \cite{Mirko2010}, etc

\subsection{Contributions}

In this paper, we

\begin{itemize}
\item connect the smooth theory of self-supporting surfaces with vertical loads to the geometry of isotropic 3-space, whose isotropic direction
is the direction of gravity \secref{sec:smooth}. In this view the stress surface $\phi(x,y)$ can
be viewed as a relative sphere of a dual relative geometry, and
one can express the characterizing equations of a 
self-supporting surface \eqref{eq:conds}
in terms of curvatures.
\item discretize the smooth theory to polyhedral thrust networks \secref{sec:discrete}, and derive an analogous equivalence between thrust networks in static equilibrium and those that satisfy a condition on their \emph{discrete} isotropic and relative curvatures.
\item present an optimization algorithm for finding a thrust network near a given arbitrary reference surface \secref{sec:opt}, and build a tool for interactive design of self-supporting surfaces based on this algorithm \secref{sec:design}.
\item exploit the relationship between a self-supporting surface and its relative sphere to find planar quadrilateral (PQ) representations of thrust networks \secref{sec:pq}, and construct particularly nice families of self-supporting surfaces \secref{sec:koenig}.
\item demonstrate with examples the versatility and applicability of our approach to the design and analysis of large-scale masonry and steel-glass structure \secref{sec:results}.
\end{itemize}

\section{Self-supporting Surfaces and Isotropic Geometry} \label{sec:smooth}
%%%%%%%%%%%%%%%%%%%%%%%%%%%%%%%%%%%%%%%%%%%%%%%%%%%%%%%%%%%%%%%%%%%%%%%%%%%%%%%%%%%%%%%%%%%%%%%%%


Here, I assume some basic knowledge of isotropic geometry, as for example described in
\cite{Pottmann2007}. Isotropic geometry possesses a metric duality. Parallel points (i.e., vertically
aligned points = points with the same top view) are dual to parallel planes. The isotropic distance
of two points is dual to the i-angle of 2 planes. 

Relative differential geometry of a surface $S$ takes a convex surface $R$ (replacing the usual unit sphere) and
defines a Gauss mapping $S \mapsto R$ via parallel tangent planes. The (negative) derivative of this
map is the relative Weingarten map and now everything is as usual: the trace equals twice the
 $r$-mean curvature $H_r$, the determinant
is the $r$-Gaussian curvature $K_r$, eigenvalues and eigenvectors give $r$-principal curvatures and
their directions, respectively.

Now we apply metric duality to this concept. We take a surface $S: z=f(x,y)$ and map it via parallel
points to a convex (or concave) relative sphere $R: z=g(x,y)$. The tangent planes of $S$, $z=f_x x+f_y y+ ...$, have
the essential plane coordinates $(f_x,f_y)$ (determining their orientation via the Euclidean normal vector $(f_x,f_y,-1)$), and likewise for $g$. Thus, the associated plane
mapping is
%
\be \nabla g \mapsto \nabla f. \ee
%
We take the other direction than in the case outlined above (where the map goes from $S$ to $R$), so that we obtain the standard isotropic curvatures for the isotropic rel. sphere $2z=x^2+y^2$.
To get the derivative of the plane mapping, we take a curve on $R$ (top view $\xw(t)$) and consider the family of tangent
planes $\nabla g(\xw(t))$. Their derivatives (orthogonal to top views of ruling vectors of generated developable
surfaces) are $\nabla^2 g  \cdot \dot{\xw}$. So the derivative mapping (corresponding to the Weingarten
map) is 
%
$$ \nabla^2g \cdot \dot{\xw} \mapsto \nabla^2f \cdot \dot{\xw}$$
%
Setting $\uw=\nabla^2g \cdot \dot{\xw}$, we finally get the Weingarten map of our dual relative geometry (as
in isotropic geometry, we do not take the negative of this map):
%
\be \uw \mapsto W^R \cdot \uw \ \ {\rm with} \ W^R=\nabla^2 f \cdot ( \nabla^2 g)^{-1}. \ee
%
The dual relative curvatures (which we will simply call $R$-curvatures) are now defined via
%
\be 2 H^R = {\rm trace}(W^R), K^R=\det(W^R). \ee
%
Eigendirections of $W^R$ determine the relative principal curvature directions. Geometrically speaking, relative principal curvature
lines form that conjugate curve network on $S$ which is vertically aligned with a conjugate
curve network on the surface $R$. 

In the special case where $R$ is the isotropic unit sphere $2z=x^2+y^2$, $\nabla^2 g$ is the identity
matrix and thus the i-mean curvature is  $2H^i =\Delta f$ and i-Gaussian curvature equals $K^i=\det(\nabla^2 f)$.

Clearly, we have
%
\be ( \nabla^2 g)^{-1}={ \frac{1}{g_{xx}g_{yy}-g_{xy}^2}}\left( \begin{array}{cc} g_{yy} & -g_{xy} \\ -g_{xy} & g_{xx} \end{array} \right)=:{\frac{1}{K^i}}M, \ee
%
where $K^i$ denotes the isotropic Gaussian curvature of the surface $R$. 
The matrix $M=(m_{ij})$ obviously has rows and columns with vanishing divergence, which is also the
 condition on
$M$ in equation (\ref{equi}). 
The trace of the matrix $W^R$ equals $2H^R$, and hence we have
%
$$ 2 K^i H^R  = m_{11}f_{xx}+2m_{12}f_{xy}+m_{22}f_{yy}. $$
%
This is the same as the left hand side of equation (\ref{equi}), if we identify the
matrices $M$ that are present in both equations. Hence, if we take as a 
relative sphere the surface $z=g(x,y)$ underlying $M$ (Airy stress surface), which has to be
a convex surface ($M$ positive semi-definite), we can express the vertical part of the equilibrium
conditions (\ref{eq:conds}) in terms of curvatures as
%
\be 2 K^i H^R = F. \label{equigeo} \ee
%
Recall: $K^i$ is the isotropic Gaussian curvature of the Airy surface $R$ and $H^R$ is the relative mean
curvature of the self supporting surface $S$ with respect to $R$. 

Let us elaborate a bit on this finding.

\begin{enumerate}
\item The Airy surface $R$ is itself brought into equilibrium by the same horizontal
forces as do $S$. From (\ref{equigeo}), because $H^R(R)=1$ the corresponding vertical loads are $F^R=2K^i$. Hence, $H^R$ can
be written in terms of the forces on $S$ and $R$ via
%
\be H^R = F/F^R. \label{equigeo2} \ee
%
\item In case of an isotropic horizontal force distribution, $M=cI, c>0$, is a scalar multiple of the identity matrix and thus the stress surface 
$R$ is the isotropic sphere $2z=c(x^2+y^2)$. We obtain with $K^i=c^2$ the Poisson equation 
$2c^2 H^R = 2 c^2 \Delta f = F$. So if in addition we impose constant forces $F$, the surfaces
are isotropic CMC surfaces.

\item All surfaces with the same stress surface and the same loads
(but different boundaries) differ just by a surface $D$ with $H^R=0$, i.e., by a relative minimal surface. 
\end{enumerate} 

\section{Discretization} \label{sec:discrete}

\subsection{Thrust Networks}

Consider now the discretization of a self-supporting surface by a polyhedral mesh $M$. We again assume the loads on $M$ are vertical, and discretize them as a force density $F_i$ assigned to each vertex $\qw_i$ of $M$. We assume that stresses are carried by the edges $\ew_{ij}$ of $M$ (with $i,j$ the indices of the vertices joined by this edge) and act only on vertices. The stress along each edge can be written as a scalar multiple $w_{ij}$ of the edge vector itself; for a given set of such weights $\ww$, a vertex $i$ is in static equilibrium if the forces are in balance:
\begin{align}
\sum_{j \sim i} w_{ij} \left(x_j - x_i\right) &= 0\notag \\
\sum_{j \sim i} w_{ij} \left(y_j - y_i\right) &= 0 \label{eq:deqtop} \\
\sum_{j \sim i} w_{ij} \left(z_j - z_i\right) &= A_i F_i, \label{eq:deqz}
\end{align}
where $A_i$ is the Voronoi area of vertex $i$ in the \emph{top view} $M'$ of $M$ (the projection of $M$ onto the $xy$ plane), and $F_i$ a force density in the top view. A mesh $M$ with associated edge weights $\ww$ is a discrete \emph{thrust network} if the balance equations \eqref{eq:deqz} holds at every interior and free boundary vertex $i$. When all the weights $w_{ij}$ are nonnegative, this thrust network is compressive.

The first pair of equations \eqref{eq:deqtop} have a geometric interpretation: if we denote by $\ew'_{ij}$ the oriented edge vector $(x_j, y_j) - (x_i, y_i)$, the equations \eqref{eq:deqtop} assert the existence of a local, orthogonal dual \emph{reciprocal diagram} to $M'$, whose edges are given by $w_{ij} \ew'^{\perp}_{ij}.$ When $M$ is simply-connected, this reciprocal diagram can be globally embedded in the plane.

\subsection{Metric Duality, Isotropic and Relative Curvatures}
For every mesh $M$ with planar faces, it is possible to construct a dual mesh $M^*$ by identifying primal faces $z = ax + by + c$ with dual vertices $(a,b,-c)$~\cite{Maxwell64}. This duality maps normal curvature at a primal edge to isotropic (top-view) length of dual edges. Since the intersection of two planes is orthogonal to both plane normals, it is orthogonal to their linear combinations, and so the top view $M'^{*}$ of the dual mesh is a recriprogal diagram of the primal top view $M'$. 

Following Pottmann and Liu~\shortcite{Pottmann2007}, this duality leads to a natural discretization of relative mean curvature of $M$ with respect to the discrete sphere $R$ at a vertex $i$ by dual areas and mixed dual areas:
\begin{align*}
H^R_i &= \frac{D(\mw_i, \rw_i)}{D(\rw_i)}\\
D(\mw_i,\rw_i) &= \frac{1}{4}\sum_{j=1}^n \left( \det(\qw'^*_j, \pw'^*_{j+1}) + \det(\qw'^*_{j+1}, \pw'^*_j)\right)\\
D(\rw_i) &= \frac{1}{2}\sum_{j=1}^n \det(\qw'^*_j, \qw'^*_{j+1}),
\end{align*}
where $\pw'^*$ are the top views of the vertices dual to the faces of $M$ neighboring primal vertex $i$, and $\qw^*$ are the top views of vertices dual to the neighboring faces on $R$.

To discretize \eqref{equigeo}, we also need a discrete notion of isotropic Gaussian curvature. In the smooth setting, isotropic Gaussian curvature of a small patch $F$ of a surface is approximated by the ratio of signed isotropic areas
\begin{equation*}
K^i(F) \approx \frac{\textrm{area}(\sigma(F))}{\textrm{area}(F)},
\end{equation*}
with exact equality in the limit as $F$ shrinks to a point, where $\textrm{area}(\sigma(F))$ is the istrotopic area of the image under the isotropic Gauss map of $F$, and $\textrm{area}(F)$ the isotropic area of $F$. This property of isotropic Gaussian curvature motivates the following definition of discrete Gaussian curvature of a mesh $R$ at a vertex $i$:
\begin{equation*}
K^i_i = \frac{D(i)}{A_i},
\end{equation*}
with $D$ the dual area as above, and $A_i$ the Voronoi area of vertex $i$ in the top view $R'$.

\subsection{Discrete Balance Equation}
We now prove the discrete analogue to equation \eqref{equigeo}.

\begin{theorem}
A simply-connected mesh $M$ can be put into static equilibrium if and only if there exists a mesh $R$ with planar faces and identical top view, $R' = M'$, satisfying
\begin{equation}
2 K_i(R) H^R_i(M) = F_i \label{eq:deqiso}
\end{equation}
at every interior and free boundary vertex $i$. $M$ can be put into compressive static equilibrium if and only if there exists a convex such $R$.
\end{theorem}
\begin{proof}
Since $R$ has planar faces, it has a well-defined polar dual; let $w_{ij} = |\rw'^{*}_{ij}|/|\rw'_{ij}|.$ These weights are positive when $R$ is convex, and satisfy the top view equilibrium equations \eqref{eq:deqtop}.

From the definition of discrete isotropic Gaussian curvature and relative mean curvature, equation \eqref{eq:deqiso} is equivalent to
\begin{equation*}
2 \frac{D(\rw_i)}{A_i} \frac{D(\mw_i, \rw_i)}{D(\rw_i)} = F_i.
\end{equation*}
It can be shown that \todo{(in the appendix?)}
\begin{equation*}
2 D(\mw_i, \rw_i) = \sum_{j\sim i} w_{ij} \left(z_j - z_i\right),
\end{equation*}
completing the proof that $M$ can be put into static equilibrium. Conversely, every simply-connected mesh in static equilibrium has a reciprocal diagram, whose polar dual by a similar argument is the desired mesh $R$ satisfying \eqref{eq:deqiso}.

\end{proof}


\section{Thrust Networks from Reference Meshes} \label{sec:opt}
Suppose we are given a reference mesh, with vertices at positions $\tilde{\qw}_i$. This mesh might be an existing structure whose stability we wish to analyze, or the output of an interactive editing operation on a surface being designed. We wish to find a nearby mesh, with identical combinatorics but different positions $\qw_i$ (subject to boundary conditions). The loads on this thrust network include user-prescribed loads as well as the self-weight of the thrust network mesh (which depends on $\qw_i$.) Conceptually, finding this thrust network amounts to minimizing $\|\qw - \tilde{\qw}\|$ subject to constraints \eqref{eq:deqtop}, \eqref{eq:deqz}, and $w_{ij} \geq 0$. This nonlinear, non-convex, inequality-constrained variational problem cannot be efficiently solved in practice, however; instead we can do so approximately, using the following algorithm:
\begin{enumerate}
\item Start with the initial guess $\qw = \tilde{\qw}$.
\item Update our estimate of $F_i$.
\item Fixing positions, find weights that come closest to bringing the thrust network to equilibrium.
\item Fixing the weights, alter positions decrease equilibrium violation.
\item Repeat from step 2 until convergence.
\end{enumerate}

\subsection{The Initial Guess} \label{sec:optinit}
If we are analyzing a reference mesh for the first time, we might as well assume it is already self-supporting, and set $\qw = \tilde{\qw}$. If we are interactively editing a reference mesh, and $\tilde{\qw}$ is an incremental update to a previous reference mesh for which we have already found a nearby thrust network, we can use the old thrust network as the initial guess.

\subsection{Update The Loads}
Since we want to include the self-weight of the surface as a load on the surface, the loads depend not only on $\Omega$ but also on thrust network surface area. To avoid adding nonlinearity to the algorithm we assume during each iteration that the loads are fixed, with self-load at vertex $i$ given by a constant surface density $\rho$ times the Voronoi area $A_i$ of vertex $i$.

\subsection{Find The Best Weights} \label{sec:optweights}
In this step, we fix positions and try to find weights that come closest to putting the thrust network in equilibrium, in the least-squares sense:
\begin{align*}
\min_{w_{ij}}\ &\sum_i \left\| \sum_{j\sim i} w_{ij} (\qw_j - \qw_i) - (0,0,F_i) \right\|^2\\
&\textrm{s.t.}\quad 0 \leq w_{ij} \leq w_{\textrm{max}},
\end{align*}
where the outer sum is over the interior and free boundary vertices $i$, and $w_\textrm{max}$ is an optional maximum weight we are willing to assign (to limit the amount of stress in the surface). This convex, sparse, bound-constrained least-squares problem always has a solution. If the objective is $0$ at this solution, we have found weights that put the thrust network in equilibrium -- we are done. Otherwise, we need to move some positions.

\subsection{Find Better Positions}
Once we have found the best possible weights for fixed thrust network positions, we fix these weights and move positions. We could just immediately solve for positions that put the thrust network in equilibrium with the given weights -- doing so involves solving a linear system that almost always has a solution. But with this approach the resulting thrust network positions are often far from the reference mesh; a better one is to look for new positions that decrease the imbalance in the stresses and loads, while also penalizing drift away from the current thrust network and reference mesh:
\begin{align*}
\min_{\qw}\ &\sum_i \left\| \sum_{j\sim i} w_{ij} (\qw_j - \qw_i) - (0,0,F_i) \right\|^2 \\
&+ \alpha \sum_i (\nw_i \cdot (\qw_i - \qw'_i) )^2 + \beta \|\qw - \qw'_P\|^2,
\end{align*}
where $\qw'_i$ is the position of the $i$-th vertex at the start of the optimization, $\nw_i$ is the starting vertex normal, $\qw'_P$ is the projection of $\qw'$ onto the reference mesh, and $\alpha > \beta$ are penalty coefficients that are decreased every iteration of steps 2-4 of the algorithm. The second term allows the thrust network to slide over itself (if doing so improves equilibrium) but penalizes normal drift. The third term, weaker than the second, penalizes drift away from the reference mesh and excessive tangential sliding.

Solving this weighted least-squares problems amounts to solving a sparse linear system.

\subsection{Convergence}
\todo{TODO Clean up, replace with more definite results}
This algorithm is not guaranteed to always converge: in fact, it is easy to set up situations where it does not: if the boundary of the reference mesh encloses too large of a region, $w_{\max}$ is set too low, and the density of the surface too high, a thrust network in equilibrium simply does not exist (the vault is too ambitious and cannot be built to stand; pillars etc. are needed.) 

Although I cannot say or prove anything precise about convergence, I can make a few remarks: step 3 always decreases the equilibrium energy $E=\sum_i \left\| \sum_{j\sim i} w_{ij} (\qw_j - \qw_i) - (0,0,F_i) \right\|^2$, and step 4 does as well as $\beta \to 0$. Moreover, as $\alpha \to 0$ and $\beta \to 0$, step 4 becomes a linear system with as many equations as unknowns; if this system has full rank, its solution sets $E=0$. These facts suggest that if step 2 is skipped, and the loads kept constant, the algorithm should \emph{usually} converge to a thrust network in equilibrium, and this is what I observe in practice. On the other hand, I know of ways to guarantee that the system in step 4 doesn't have full rank -- setting all edge weights around a vertex (or, more generally, an interior region of the mesh) to 0, for instance.

Step 2 also almost always increases $E$, by changing the loads. It would be nice if I could prove that, for sufficiently large $w_{\textrm{max}}$, step 2 cannot cause the algorithm to fail to converge, but I don't yet know how to approach this problem.

\section{Design of Self-Supporting Surface} \label{sec:design}
The optimization algorithm described in the previous section forms the basis of an interactive design tool for self-supporting surfaces. Users manipulate a mesh representing a reference surface, and the computer searches for a nearby thrust network in equilibrium. Fitting this thrust network does not require that the user specify boundary tractions, and although the top view of the reference mesh is used as an initial guess for the top view of the thrust network \secref{sec:optinit}, the search is not restricted to this top view.

Some features of the design tool include:
\begin{itemize}
\item Handle-based 3D editing of the reference mesh using Laplacian coordinates~\cite{Lipman2004,Sorkine2003} to extrude vaults, insert pillars, and apply other deformations to the reference mesh. Handle-based adjustments of the heights, keeping the top view fixed, and deformation of the top view, keeping the heights fixed, are also supported. The thrust network adjusts interactively to fit the deformed positions, giving the usual visual feedback about the effects of her edits on whether or not the surface can stand.
\item Specification of boundary conditions. Points of contact between the reference surface and the ground or environment are specified by ``pinning'' vertices of the surface, specifying that the thrust network must coincide with the reference mesh at this point, and relaxing the condition that forces must be in equilibrium there. 
\item Interactive adjustment of surface density $\rho$, external loads, and maximum permissible stress per edge $w_{\textrm{max}}$ \secref{sec:optweights}, with visual feedback of how these parameters affect the fitted thrust network.
\item Upsampling of the thrust network through Catmull-Clark subdivision~\cite{TODO} and polishing of the resulting refined thrust network using optimization \secref{sec:opt}.
\item Visualization of the stress surface $R$ dual to the thrust network.
\end{itemize}
\todo{TODO more to come?}
\subsection{Limitations}
\todo{TODO Limits on the size of the mesh due to performance; cases where the optimization breaks down (bad reference geometry), etc}

\section{Self-supporting PQ Meshes} \label{sec:pq}
Meshes whose faces are planar are of particular interest in architecture. Quadrilateral meshes are not generally planar, nor is a typical thrust network with quadrilateral top view. In this section we propose an algorithm for remeshing a thrust network in equilibrium so that its faces are planar quads (PQ).

Consider a thrust network $M$ in equilibrium, with its corresponding discrete stress surface $R$. The top views of $M$ and $R$ coincide. The edges of PQ meshes are discretizations of conjugate curve networks over smooth surfaces~\cite{Liu2006} -- pairs of families of curves that are orthogonal with respect to the inner product induced by the second fundamental form. This fact suggests searching for a new top view for $M$ whose edges are the projections of a conjugate curve network onto the plane. But since $M$ and $R$ share a top view, and $R$ must be planar for $M$ to be in equilibrium, this new top view must \emph{also} be the projection of a conjugate curve network on $R$.

We first characterize such networks in the smooth setting. Let $m(x,y)$ and $r(x,y)$ be a self-supporting surface and its stress surface, and $\uw(x,y)$ a vector field of principle relative curvature directions in $\mathbb{R}^2$: eigenvectors of the Weingarten map $(\nabla^2 r)^{-1}\nabla^2 m$. Then
\begin{align*}
(\nabla^2 r)^{-1}\nabla^2 m \uw &= \lambda \uw\\
\nabla^2 m \uw &= \lambda \nabla^2 r \uw.
\end{align*}
In particular, if $\vw(x,y)$ is orthogonal to $\uw$ in the plane with respect to the metric $\nabla^2 m$, then
\begin{equation*}
0 = \vw^T \frac{\nabla^2 m}{\sqrt{1+m_x^2+m_y^2}} \uw = \vw^T \frac{\nabla^2 r}{\sqrt{1+r_x^2+r_y^2}} \uw.
\end{equation*}
These last two are the second fundamental forms of $m$ and $r$, so we have a conjugate curve network on each surface, whose top views are identical. The first family of curves is given by $(\uw, \uw\cdot \nabla m)$ on $m$, $(\uw, \uw\cdot \nabla r)$ on $r$, and similarly for the family generated by $\vw$.

In the discrete setting, remeshing $M$ and $R$ using a top view whose edges approximate integral curves of $\uw$ and $\vw$, using for instance the techniques of Schiftner~\shortcite{Schiftner2007} and Cohen-Steiner and Morvan~\shortcite{Cohen-Steiner2003} generates new $M$ and $R$ with near-PQ faces, since the edges of $M$ and $R$ closesly approximate conjugate curves on these surfaces. These surfaces can then be polished to PQ meshes using optimization.
\todo{TODO try this}

\section{Results} \label{sec:results}

\subsection{Concrete Vaults}

\subsection{PQ Steel-Glass Structures}

\subsection{Koenig Mesh} \label{sec:koenig}

\section{Conclusion and Future Work}

\todo{TODO}

\section*{Acknowledgements}

\bibliographystyle{acmsiggraph}
\bibliography{selfsupporting}

\section{Appendix}
\subsection{Proof of Mixed-Area Formula}
Let $M$ and $R$ be meshes with identical top views $M'=R'$. Then for every vertex $i$,
\begin{equation*}
2D(\mw_i,\rw_i) = \sum_{j\sim i} w_{ij} (z_j - z_i),
\end{equation*}
where $z_j$ is the height of vertex $j$ in $M$, and $w_{ij}$ is the ratio of dual to primal signed isotropic edge lengths in $R$: 
\begin{equation*}
\rw'^{*}_{ij} = w_{ij}\rw'^{\perp}_{ij} = w_{ij}\mw'^{\perp}_{ij},
\end{equation*}
where $\vw^{\perp}$ is the rotation of $\vw$ clockwise by ninety degrees.

For convenience, let $\pw_1, \ldots, \pw_n$ be the points dual to the faces around vertex $i$ in $M$, and $\qw_1, \ldots \qw_n$ the corresponding points in $R^{*}$. Let $\ew'_1, \ldots \ew'_n$ be the top view of the edges from $i$, with edges $j, j+1$ neighboring face $j$. Then, with indices understood modulo $n$,
\begin{equation*}
2D(\mw_i, \rw_i) = \frac{1}{2}\sum_{j=1}^n (\det(\qw'_j \times \pw'_{j+1}) + \det(\qw'_{j+1}\times \pw'_j)).
\end{equation*}
We have $\qw'_{j+1} = \qw'_j + w_{j+1} \ew'^{\perp}_{j+1}.$ We can derive a formula for $\pw'_{j}$ by looking at the gradient of the corresponding face of $M$. On that face, if $z_0$ is the $z$-coordinates of vertex $i$ of $M$, and $z_j$ the height of the vertex at the end of edge $\ew'_j$, 
\begin{align*}
\nabla z(x,y) &= \left[\begin{array}{cc} z_j - z_0 & z_{j+1} - z_0\end{array}\right] \left[\begin{array}{cc} \ew'_{j} & \ew'_{j+1}\end{array}\right]^{-1} \\
\pw'_j &= \frac{1}{\det(\ew'_j, \ew'_{j+1})} \left( -\ew'^{\perp}_{j+1} (z_j - z_0) + \ew'^{\perp}_j (z_{j+1}-z_0) \right).
\end{align*}
Thus, using the fact that $\pw'_j$ and $\qw'_j$ and parallel,
\begin{align*}
2D(\mw_i, \rw_i) &= \frac{1}{2}\sum_{j=1}^n (\det(\qw'_j, \pw'_{j+1}) + \det(\qw'_{j+1}, \pw'_j))\\
&= \frac{1}{2}\sum_{j=1}^n \det((\qw'_{j+1} - w_{j+1} \ew'^{\perp}_{j+1}), \pw'_{j+1}) \\
&\quad + \frac{1}{2}\sum_{j=1}^n \det((\qw'_j + w_{j+1} \ew'^{\perp}_{j+1}), \pw'_j)\\
&= \frac{1}{2}\sum_{j=1}^n \left( \det(- w_{j+1} \ew'^{\perp}_{j+1}, \pw'_{j+1}) + \det(w_{j+1} \ew'^{\perp}_{j+1}, \pw'_j)\right) \\
&= \sum_{j=1}^n w_{j+1} (z_{j+1} -z_0),
\end{align*}
as desired.

\end{document}

% Work into limitations?
\subsubsection{Possible pitfall}
Though the above optimization problem always has a solution, there is one unpleasant situation can arise: the optimum weights around \emph{all} edges adjacent to a vertex might be 0. I've observed this occur when the reference mesh has an interior local minimum at vertex $i$: obviously such a reference mesh cannot be a thrust network in equilibrium. If the neighbors of $i$ have few edges (such as in a quad mesh) the optimal weights for the edges around $i$ will be positive, since these edges are "needed" by $i$'s neighbors to put them in equilibrium. On the other hand, if $i$'s neighbors are edge-rich (such as in a triangle mesh) occasionally the best solution is to set all of $i$'s edges to have weight 0, effectively deleting vertex $i$ from the thrust network. A vertex whose surrounding edge weights are all zero (or near-zero) clearly can never satisfy \eqref{eq:dcond}, and giving the next step of the algorithm such a vertex as input causes numerical instabilities. Therefore in the current code I simply delete such vertices from the thrust network.

Reference meshes with large flat regions similarly cause numerical problems: again, in such regions no amount of altering the weights can improve the error in equation \eqref{eq:dcond}, leading the above optimization to sometimes assign many near-zero weights in such regions.


