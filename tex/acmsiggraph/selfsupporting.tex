%%% The ``\documentclass'' command has one parameter, based on the kind of
%%% document you are preparing.
%%%
%%% [annual] - Technical paper accepted for presentation at the ACM SIGGRAPH
%%%   or SIGGRAPH Asia annual conference.
%%% [sponsored] - Short or full-length technical paper accepted for
%%%   presentation at an event sponsored by ACM SIGGRAPH
%%%   (but not the annual conference Technical Papers program).
%%% [abstract] - A one-page abstract of your accepted content
%%%   (Technical Sketches, Posters, Emerging Technologies, etc.).
%%%   Content greater than one page in length should use the "[sponsored]"
%%%   parameter.
%%% [preprint] - A preprint version of your final content.
%%% [review] - A technical paper submitted for review. Includes line
%%%   numbers and anonymization of author and affiliation information.

\documentclass[annual]{acmsiggraph}

\usepackage{par06}
\usepackage{amsmath}
\usepackage{amssymb}
\usepackage{amsthm}
\usepackage{overpic}
\usepackage{contour}\contourlength{1pt}
\usepackage{color}

\long\def\nix#1{\relax}


\def\div{\DID YOU RELLY WANT TO USE \div?}
\def\<{\mathchoice{\big\langle}{\langle}{\langle}{\langle}}
\def\>{\mathchoice{\big\rangle}{\rangle}{\rangle}{\rangle}}
%\def\lll{\mathopen{\mbox{$\<\hskip-.5ex\<$}}}
%\def\rrr{\mathclose{\mbox{$\>\hskip-.5ex\>$}}}
\def\wh{\widehat}
%\def\II{\mbox{I\hskip-0.1exI}}
%\def\pu{{\partial\over\partial u}}
%\def\du{{d\over d u}}
%\def\hinauf#1#2{{\mathop{#1}\limits^{\vbox to-.32564ex{\kern-.4652ex
%   \hbox{\normalfont #2}\vss}}}}
%\def\dddot#1{\hinauf{#1}{...}}
%\def\ddddot#1{\hinauf{#1}{....}}
%\def\str{/\penalty10000\hskip0pt}
%\def\d#1dot#2{\dddddot{#2}}
\newtheorem{theorem}{Theorem}
\newtheorem{prop}[theorem]{Proposition}
\def\Div{\mathop{{\rm div}}\nolimits}
\def\tr{\mathop{{\rm tr}}\nolimits}
\def\rel{{\mathord{\text{\rm rel}}}}
\def\const{{\mathord{\textrm{const}}}}
\def\ess{s}
\def\Hess#1{{\def\testess{#1}\nabla^2\ifx\testess\ess\!s\else #1\fi}}
\def\Hess#1{\text{$\nabla^2\hskip-.2ex #1$}}
\def\Forcevector{\Big(\mbox{\scriptsize
	\def\arraystretch{0.8}\begin{tabular}{@{\,}c@{\,}}
	0 \\ 0 \\ $A_i F_i$
	\end{tabular}}\Big)}
\hyphenation{pa-rab-ol-loid}

\def\lput(#1,#2)#3{\put(#1,#2){\hbox to 0pt{\hss{#3}}}}
\def\cput(#1,#2)#3{\put(#1,#2){\hbox to 0pt{\hss{#3}\hss}}}
\definecolor{blau}{rgb}{0.15,0.2,0.3}
%\definecolor{rot}{rgb}{0.7,0.5,0.2}
\definecolor{drot}{rgb}{0.7,0,0.1}
\definecolor{grey}{rgb}{0.6,0.6,0.6}
\definecolor{lightgrey}{rgb}{0.8,0.8,0.8}
\definecolor{gelb}{rgb}{.55,.40,.1}


\outer\def\proclaim #1. #2\par{\noindent{\bf#1.\enspace}{\it#2\par}}


%\usepackage{mathrsfs}
\def\SS{{\mathcal S}}
\def\RR{{\mathcal R}}


\newcommand{\todo}[1]{\textcolor{red}{#1}}
\newcommand{\secref}[1]{(\S\ref{#1})}

%%% If you are submitting your paper to one of our annual conferences - the
%%% ACM SIGGRAPH conference held in North America, or the SIGGRAPH Asia
%%% conference held in Southeast Asia - there are several commands you should
%%% consider using in the preparation of your document.

%%% 1. ``\TOGonlineID''
%%% When you submit your paper for review, please use the ``\TOGonlineID''
%%% command to include the online ID value assigned to your paper by the
%%% submission management system. Replace '45678' with the value you were
%%% assigned.

\TOGonlineid{0043}

%%% 2. ``\TOGvolume'' and ``\TOGnumber''
%%% If you are preparing a preprint of your accepted paper, and your paper
%%% will be published in an issue of the ACM ``Transactions on Graphics''
%%% journal, replace the ``0'' values in the commands below with the correct
%%% volume and number values for that issue - you'll get them before your
%%% final paper is due.

\TOGvolume{0}
\TOGnumber{0}

%%% 3. ``TOGarticleDOI''
%%% The ``TOGarticleDOI'' command accepts the DOI information provided to you
%%% during production, and which makes up the URLs which identifies the ACM
%%% article page and direct PDF link in the ACM Digital Library.
%%% Replace ``1111111.2222222'' with the values you are given.

\TOGarticleDOI{1111111.2222222}

%%% 4. ``\TOGprojectURL'', ``\TOGvideoURL'', ``\TOGdataURL'', ``\TOGcodeURL''
%%% If you would like to include links to personal repositories for auxiliary
%%% material related your research contribution, you may use one or more of
%%% these commands to define an appropriate URL. The ``\TOGlinkslist'' command
%%% found just before the first section of your document will add hyperlinked
%%% icons to your document, in addition to hyperlinked icons which point to
%%% the ACM Digital Library article page and the ACM Digital Library-held PDF.

\TOGprojectURL{}
\TOGvideoURL{}
\TOGdataURL{}
\TOGcodeURL{}

%%% Replace ``PAPER TEMPLATE TITLE'' with the title of your paper or abstract.

\title{Design of Self-supporting Surfaces}

%%% The ``\author{}'' command takes the names and affiliations of each of the
%%% authors of your paper or abstract. The ``\thanks{}'' command takes the
%%% contact information for each author.
%%% For multiple authors, separate each author's information by the ``\and''
%%% command.

%\author{Roy G. Biv\thanks{e-mail: roy.g.biv@aol.com}\\ Starbucks Research %
%\and Ed Grimley\thanks{e-mail:ed.grimley@aol.com}\\Nigel Mansell\thanks{nigelf1@msn.com}\\ Grimley Widgets, Inc. %
%\and Martha Stewart\thanks{e-mail:martha.stewart@marthastewart.com}\\ Martha Stewart Enterprises \\ Microsoft Research}

%%% The ``pdfauthor'' command accepts the authors of the work,
%%% comma-delimited, and adds this information to the PDF metadata.

\pdfauthor{Anonymous}

%%% Keywords that describe your work. The ``\keywordlist'' command will print
%%% them out.

\keywords{Self\dash supporting masonry structures, thrust networks,
reciprocal force diagrams, discrete Laplace operators, 
isotropic geometry, mean curvature}

%%% The ``\begin{document}'' command is the start of the document.

%%% If you have user-defined macros, you may include them here.

% example of a user-defined macro called ``remark.''
% \newcommand{\remark}[1]{\textcolor{red}{#1}}

\begin{document}

%%% A ``teaser'' image appears under the title and affiliation information,
%%% horizontally centered, and above the two columns of text. This is OPTIONAL.
%%% If you choose to have a ``teaser'' image, it needs to be placed between
%%% ``\begin{document}'' and ``\maketitle.''

%\teaser{
%   \includegraphics[height=1.5in]{images/sampleteaser}
%   \caption{Spring Training 2009, Peoria, AZ.}
%}

%%% The ``\maketitle'' command must appear after ``\begin{document}'' and,
%%% if you have one, after the definition of your ``teaser'' image, and
%%% before the first ``\section'' command.

\maketitle

%%% Your paper's abstract goes in its own section.

\begin{abstract}
Self\dash supporting masonry is one of the most ancient and at the same
time most elegant ways of building curved shapes, with their analysis and
modeling being a topic of geometry processing
rather than classical continuum mechanics, because of the very geometric
nature of failure of such structures. In this paper we use the
thrust network method of analysis and present 
an iterative nonlinear optimization algorithm for efficiently approximating freeform shapes
by self\dash supporting ones. This provides an interactive
modeling tool for such shapes. The rich geometry of thrust networks which
was first studied by Maxwell in the 1860s leads us to identify new viewpoints
of discrete differential geometry: we find close connections between different
objects such as a finite\dash element discretization of the Airy stress
potential, perfect graph Laplacians, and computing admissible loads
via curvatures of polyhedral surfaces.

\end{abstract}

%%% ACM Computing Review (CR) categories.
%%% See <http://www.acm.org/class/1998/> for details.
%%% The ``\CRcat'' command takes four arguments.

\begin{CRcatlist}
  \CRcat{I.3.5}{Computer Graphics}{Computational Geometry and Object Modeling}{Curve, surface, solid, and object representations};
\end{CRcatlist}

%%% The ``\keywordlist'' command prints out the keywords.

\keywordlist

%%% The ``\TOGlinkslist'' command will insert hyperlinked icon(s) to your
%%% paper. This includes, at a minimum, hyperlinked icons to the ACM article
%%% page and the ACM Digital Library-held PDF. If you added URLs to
%%% ``\TOGprojectURL'' or the other, similar commands, they will be added to
%%% the list of icons.
%%% Note: this functionality only works for annual-conference papers.

\TOGlinkslist

%%% The ``\copyrightspace'' command
%%% Do not remove this command.

\copyrightspace

%%% This is the first section of the body of your paper.

\section{Introduction}

%\todo{TODO gentle introduction to self-supporting surfaces, assumptions about the model, importance to architecture etc}


Vaulted masonry structures are among the simplest and at the same time
most elegant solutions for creating curved shapes in building
construction. This is the reason why they have been an object of interest
since antiquity, large non\dash convex examples being provided by gothic
cathedrals. They continue to be an active topic of research in today's
engineering community.


Our paper is concerned with a combined geometry+statics analysis of {\em
self\dash supporting} masonry and with tools for the interactive modeling
of freeform self\dash supporting structures. Here `self\dash supporting'
means that the structure, considered as an arrangement of blocks (bricks,
stones), holds together by itself, and additional support, additional
chains and similar are present only during construction. Our analysis is
based on the following assumptions, which follow the classic
\cite{Heyman66}:


{\it Assumption 1:} Masonry has no tensile strength, but the individual
building blocks do not slip against each other (because of friction or
mortar). On the other hand, their compressive strength is sufficiently
high so that failure of the structure is by a sudden change in geometry,
such as shown by Figure \ref{fig:block}, and not by material failure.

{\it Assumption 2 (The Safe Theorem)}: If a system of forces can be found
which is in equilibrium with the load on the structure and which is
contained within the masonry envelope then the structure will carry the
loads, although the actually occurring forces may not be those postulated.

Our approach is twofold: We first give an overview of the continuous case of a
smooth surface under stress which turns out to be governed by the so\dash
called Airy stress function, at least locally.
This mathematical model is called a
membrane in the engineering literature and has been applied to the
analysis of masonry before. The surface is self\dash supporting if and
only if stresses are entirely compressive (i.e., the Airy function is
convex). For computational purposes, stresses are discretized as a
fictitious {\em thrust network} \cite{Block07}
contained in the masonry structure.
This is a system of forces which together with the structure's deadload
is in equilibrium. It can be interpreted as a finite element discretization
of the continuous case, and it turns out to have very interesting geometry
dating back to the work of J.\ C.\ Maxwell \shortcite{Maxwell64}, with the
Airy stress function becoming a discrete polyhdron directly
related to a reciprocal force diagram. 
Our own contributions are the following:


\begin{figure}[t]
\includegraphics[width=\columnwidth]{fig/cheesevault25.jpg}
\caption{A surface with many, irregularly placed holes almost never stands
by itself; those that do are surprising and their stability is not obvious
by inspection. The surface shown is produced by out algorithm which finds,
for a given freeform shape, the nearest self\dash supporting surface.
This procedure sometimes requires large deformations, as seen for
this example in Figure~\ref{fig:cheese2} below.}
	\label{fig:cheese}
\end{figure}




\paragraph{Contributions.} 

\begin{list}{$\bullet$}{\leftmargin0pt\itemindent1em}

\item
We connect the physics of self\dash supporting surfaces with
vertical loads to the geometry of isotropic 3\dash space, whith the
direction of gravity as the distinguished direction
\secref{sec:smooth}. Taking the convex Airy potential as
unit sphere, one can express the equations  
governing self\dash supporting surfaces in terms of curvatures.


\item We employ Maxwell's construction of polyhedral thrust networks
and their reciprocal diagrams \secref{sec:discrete}, and give
an interpretation of the equilibrium conditions in terms of
discrete curvatures (like the Airy function, this global construction
works for simply connected connectivity, but is easily extended to
the general case).

\item The graph Laplacian derived from a thrust network with compressive
forces is a `perfect' one. We show how it appears in the analysis and
establish a connection with mean curvatures which are otherwise defined
for polyhedral surfaces.

\item We present an optimization algorithm for efficiently finding
a thrust network near a given arbitrary reference surface \secref{sec:opt},
and build a tool for interactive design of self\dash supporting surfaces based
on this algorithm \secref{sec:design}.

\item We exploit the geometric relationships between a self\dash
supporting surface and the `unit sphere' stress potential in order
to find  particularly nice families of self\dash supporting surfaces,
especially planar quadrilateral representations  of thrust networks
\secref{sec:special}.

\item We demonstrate the versatility and applicability of
our approach to the design and analysis of large\dash scale masonry and
steel\dash glass structures.

\end{list}

	\begin{figure}[t]
\nix{
	\includegraphics[height=.35\columnwidth]{fotos/gothicvaults1}\hfill
	\includegraphics[height=.35\columnwidth]{fotos/Frauenkirche_um_1897}
	\hfill
	\hfill
	\begin{minipage}[b]{.37\columnwidth} \caption{Nonconvex self\dash
supporting masonry. Left: vaults of a gothic cathedral; Right: dome of
Frauenkirche, Dres\-den}
	\label{fig:tholos}
	\end{minipage}
	\bigskip}
\includegraphics[height=.245\columnwidth]{fotos/block1.jpg}\hfill
\includegraphics[height=.245\columnwidth]{fotos/block2.jpg}
	\caption{Masonry fails via geometric
catastrophe rather than material failure (models by 
Block Research Group, ETH Z\"urich).}
	\label{fig:block}
	\end{figure}


\paragraph{Related Work.}

Unsupported masonry has been an active topic of research in the 
engineering community. The foundations for the modern approach were laid 
by Jacques Heyman \shortcite{Heyman66} and are available as the textbook 
\cite{Heyman95}. A unifying view on polyhedral surfaces, compressive 
forces and corresponding `convex' force diagrams is presented by 
\cite{Ash1988}. F.~Fraternali \shortcite{Fraternali2002a}, 
\shortcite{Fraternali2010} established a connection between the continuous 
theory of stresses in membranes and the discrete theory of forces in 
thrust networks, by interpreting the latter as a certain non-conforming finite 
element discretization of the former.

Several authors have studied the problem of finding discrete compressive 
force networks contained within the boundary of masonry structures; early 
work in this area includes \ \cite{schek74}, \cite{Livesley92}, and\ 
\cite{O'Dwyer98}. Fraternali~\shortcite{Fraternali2010} proposed solving 
for the structure's discrete stress surface, and examining its convex hull 
to study the structure's stability and susceptibility to cracking. 
Philippe Block's seminal thesis introduced the method 
of {\it Thrust Network Analysis}, which linearizes the form-finding problem by 
first seeking a reciprocal diagram of the top view, which guarantees 
equilibrium of horizontal forces, then solving for the heights that 
balance the vertical loads (see e.g.\ \cite{Block07,block09}). Recent work by Block and coauthors extends this 
method in the case where the reciprocal diagram is not unique; for 
different choices of reciprocal diagram, the optimal heights can be found 
using the method of least squares~\cite{vanmele2011}, and the search for 
the best such reciprocal diagram can be automated using a genetic 
algorithm~\cite{Block2011}.

Other approaches to the interactive design of self-supporting structures 
include modeling these structures as damped particle-spring 
systems~\cite{Kilian2005,barnes09}, and mirroring the rich tradition in architecture of 
designing self-supporting surfaces using hanging chain 
models~\cite{Heyman98}. Alternatively, masonry structures can be 
represented by networks of rigid blocks~\cite{Whiting09}, whose conditions 
on the structural feasibility were incorporating into procedural modeling 
of buildings.

Algorithmic and mathematical methods relevant to this paper are work on 
the geometry of quad meshes with planar faces (\cite{Glymph2004}, 
\cite{Liu2006}), discrete curvatures for such meshes
\cite{Pottmann2007b,bobenko-2010-ct}, 
in particular curvatures in isotropic geometry \cite{Pottmann2007}. 
Schiftner and Balzer \shortcite{Schiftner2010} discuss approximating a 
reference surface by quad mesh with planar faces, whose layout is guided 
by statics properties of that surface.

%\cite{Koenderink2002}, etc PQ meshes

\section{Self-supporting Surfaces}

\subsection{The Continuous Theory}

We are here modeling masonry as a surface $S$ given by a height field
$s(x,y)$ defined in some planar domain $\Omega$. We assume that there are
vertical loads $F(x,y)$ --- usually $F$ represents the structure's own
weight. By definition this surface is self\dash supporting, if and only if
there exists a field of compressive stresses which are in equilibrium with
the acting forces. This is equivalent to existence of a field $M(x,y)$ of
$2\times 2$ symmetric positive semidefinite matrices satisfying
	\begin{align}
	\Div (M\nabla s) = F, \quad
	\Div M &= 0,
	  \label{eq:conds}
	\end{align}
 where the divergence operator $\Div{u(x,y)\choose v(x,y}= u_x + v_y$ is
understood to act on the columns of a matrix (see e.g.\
\cite{Fraternali2010}, \cite{Giaquinta1985}). 

The condition $\Div M=0$ says that $M$ is essentially the Hessian of a
real\dash valued function $\phi$ (the {\em Airy stress potential}): With
the notation
	$$
	M =
	{\textstyle {m_{11} \ m_{12} \choose m_{12} \ m_{22}}}
	\iff	
	\wh M =
	{\textstyle {\hphantom{-}m_{22} \ -m_{12} \choose -m_{12}
		 \ \hphantom{-}m_{11}}}
	$$
 it is clear that $\Div M=0$ is an integrability condition for $\wh M$, so
locally there is a potential $\phi$ with
	$$
	\wh M = \Hess\phi, \quad \text{i.e.,}\quad
	M = \wh{\Hess\phi}.
	$$
 If the domain $\Omega$ is simply connected, this relation holds globally.
Positive semidefiniteness of $M$ (or equivalently of $\wh M$) 
characterizes {\em convexity} of the Airy potential $\phi$.
The Airy function enters computations only by way of its derivatives,
so global existence is not an issue.

{\it Remark:} Stresses at boundary points depend on the way the surface is
anchored: A fixed anchor means no condition, but a free boundary with
outer normal vector $\nw$ means $\<M \nabla s, \nw \> = 0$.


\paragraph{The stress Laplacian.}
Note that $\Div M =0$ yields $\Div(M\nabla s)$ $ =$ $ \tr(M\Hess
s)$, which we like to call $\Delta_\phi s$. The operator $\Delta_\phi$ is
symmetric. It is elliptic (as a Laplace operator should be) if and
only if $M$ is positive definite, i.e., $\phi$ is strictly convex. The
balance condition \eqref{eq:conds} may be written as
	$
	\Delta_\phi s = F.
	$


\subsection{Discrete Theory: Thrust Networks}

We are discretizing a self-supporting surface by a polyhedral
mesh $\SS=(V,E,F)$ (see Figure~\ref{fig:reciprocal}). Loads are again
vertical, and we discretize them as force densities $F_i$ associated with
vertices $\vw_i$. The load acting on this vertex is then given by
$F_iA_i$, where $A_i$ is an area of influence (using a prime to indicate
projection onto the $xy$ plane, $A_i$ is the area of the Voronoi cell of
$\vw_i'$ w.r.t.\ $V'$). We assume that stresses are carried by the edges
of the mesh: the force exerted on the vertex $\vw_i$ by the edge
connecting $\vw_i,\vw_j$ is given by
	$$
	w_{ij} (\vw_j-\vw_i),
	\quad
	\text{where}\quad
	w_{ij}=w_{ji}\ge 0.
	$$
 The nonnegativity of the individual weights $w_{ij}$ expresses the
compressive nature of forces. The balance conditions at vertices then read
as follows: With $\vw_i=(x_i,y_i,s_i)$ we have
	\begin{align}
	\sum\nolimits_{j\sim i}
		w_{ij} (x_j - x_i)
	=
	\sum\nolimits_{j\sim i}
		w_{ij} (y_j - y_i) &= 0,
			 \label{eq:deqtop} \\
	\sum\nolimits_{j\sim i}
		w_{ij} (s_j - s_i)
		&= A_i F_i.
			\label{eq:deqz}
	\end{align}
 A mesh equipped with edge weights in this way is a discrete \emph{thrust
network}. Invoking the safe theorem, we can state that a masonry structure
is self\dash supporting, if we can find a thrust network with compressive
forces which is entirely contained within the structure.

  \begin{figure}[t]
  \centering
  \begin{overpic}[width=.94\columnwidth]{fig/reciprocal}
	\put(0,33){$\SS$}
	\lput(13,33){$\vw_i$}
	\cput(14,25){\contour{white}{$A_iF_i$}}
	\color{gelb}
	\lput(100,19){$\SS'^*$}
	\color{blau}
	\put(0,9){$\SS'$}
	\color{drot}
	\put(1,0){$w_{ij} \ew_{ij}'$}
	\lput(63,3){$\ew_{ij}^*$}
  \end{overpic}\nix{
 \begin{overpic}[width=\columnwidth]{fig/reciprocal1}
        \color{gelb}
        \lput(100,15){$\SS'^*$}
        \color{blau}
        \put(1,25){$\SS$}
        \lput(7,1){$\SS'$}
        \lput(21,35){$\vw_i$}
        \cput(21,29){$F_i$}
        \color{drot}
        \lput(13,8){\contour{white}{$w_{ij} \ew_{ij}'$}}
        \lput(66,11){\contour{white}{$\ew_{ij}^*$}}
  \end{overpic}}\relax
 \caption{A thrust network $\SS$, with dangling edges indicating
external forces (left). This network
together with compressive forces which balance vertical loads
$A_iF_i$ projects
onto a planar mesh $\SS'$ with equilibrium compressive forces
$w_{ij}\ew_{ij}'$ in its edges.
Rotating forces by 90$^\circ$ leads to the reciprocal force diagram
$\SS'^*$ (right).}
  \label{fig:reciprocal}
  \end{figure}

\paragraph{Reciprocal Diagram.}

Equations \eqref{eq:deqtop} have a geometric interpretation: With edge
vectors
	$$\ew'_{ij} = \vw_j'-\vw_i'=(x_j, y_j) - (x_i, y_i),
	$$
 Equation \eqref{eq:deqtop} asserts that vectors $w_{ij} \ew_{ij}'$ form a
closed cycle. Rotating them by 90 degrees, we see that likewise
	$$
	\ew_{ij}^{\prime *} = w_{ij} J \ew_{ij}', \quad \text{with}\quad
	J={\textstyle{0 \ -1 \choose 1 \ \hphantom{-}0}}.
	$$
 form a closed cycle (see Figure \ref{fig:reciprocal}).
If the mesh $\SS$ is simply connected, there exists
an entire {\em reciprocal diagram} $\SS^{\prime *}$ which is a
combinatorial dual of $\SS$, and which has edge vectors $\ew_{ij}'^*$.
 Its vertices are denoted by $\vw_i^{\prime *}$. 

\paragraph{Polyhedral Stress Potential.}

We can go further and construct a convex polyhedral `Airy stress potential'
surface $\Phi$ with
vertices $\ww_i=(x_i,y_i,\phi_i)$ combinatorially equivalent to $\SS$ by
requiring that a primal face of $\Phi$ lies in the plane $z=\alpha x +
\beta y + \gamma$ if and only if $(\alpha,\beta)$ is the corresponding
dual vertex of $\SS'^*$ (see Figure~\ref{fig:polarity}). Obviously this
condition determines $\Phi$ up to vertical translation. For existence see
\cite{Ash1988}. The inverse procedure constructs a reciprocal diagram from
$\Phi$. This procedure obviously works also if forces are not compressive:
we can construct an Airy mesh $\Phi$ which has planar faces, but it will
no longer be a convex polyhedron.

The vertices of $\Phi$ can be interpolated by a piecewise\dash linear
function $\phi(x,y)$. It is easy to see that the derivative of $\phi(x,y)$
jumps by the amount $\|\ew_{ij}'^*\| = w_{ij}\|\ew_{ij}'\|$, when crossing
over the edge $\ew'_{ij}$ at right angle, with unit speed. This identifies
$\Phi$ as the Airy polyhedron introduced by \cite{Fraternali2002a} as a
finite element discretization of the continuous Airy function (see also
\cite{Fraternali2010}).

If the mesh is not simply connected, the reciprocal
diagram and the Airy polyhedron exist only locally, unique up to vertical
tarnslation.

  \begin{figure}[t]
 \centerline{\vphantom{\includegraphics[width=.94\columnwidth]
		{fig/reciprocal}}\relax
  \begin{overpic}[width=.94\columnwidth]{fig/beide}
	\lput(49,33){$\ww_k^*$}
	\lput(49,15){$\vw_k^{*\prime}$}
	\color{gelb}
	\put(86,10){$\SS'^*=\Phi^{*\prime}$}
	\put(83,30){$\Phi^*$}
	\color{blau}
	\cput(24,35){$\Phi$}
	\put(0,0){$\Phi'=\SS'$}
	\cput(4,31){$f_k$}
  \end{overpic}}
%  \begin{overpic}[width=.94\columnwidth]{fig/beide1}
%        \put(46,30){$\ww_k^*$}
%        \put(46,16){$\vw_k^{*\prime}$}
%        \color{gelb}
%        \put(84,10){$\Sw'^*=\Phi^{*\prime}$}
%        \put(78,30){$\Phi^*$}
%        \color{blau}
%        \cput(30,30){$\Phi$}
%        \lput(8,1){$\Phi'=\SS'$}
%        \cput(4,33){$f_k$}
%  \end{overpic}
 \caption{Airy stress potential $\Phi$ and its polar dual $\Phi^*$.
$\Phi$ projects onto the same planar mesh as $\SS$ does, while
$\Phi^*$ projects onto the reciprocal force diagram.  A primal face
$f_k$ lies in the plane $z=\alpha x + \beta y + \gamma$ $\iff$
the corresponding dual vertex is $\ww_k^*=(\alpha,\beta,-\gamma)$.}
  \label{fig:polarity}
  \end{figure}


\paragraph{Polarity.}

Polarity with respect to the {\em Maxwell paraboloid} $z={1\over 2}
(x^2+y^2)$ maps the plane $z=\alpha x + \beta y + \gamma$ to the point
$(\alpha,\beta,-\gamma)$. Thus, applying polarity to $\Phi$ and projecting
the result $\Phi^*$ into the $xy$ plane reconstructs the reciprocal
diagram $\Phi^{*\prime}=\SS^{\prime *}$ (see Fig.~\ref{fig:polarity}).

\paragraph{Discrete Stress Laplacian.}

The weights $w_{ij}$ may be used to define a graph
Laplacian $\Delta_\phi$ which on vertex\dash based functions acts as
	$$\Delta_{\phi} s(\vw_i)=\sum\nolimits_{j\sim i} w_{ij}(s_j-s_i).$$
 This operator is a perfect discrete Laplacian in the sense
of \cite{wardetzky07}, since it is symmetric
by construction, Equation \eqref{eq:deqtop} implies linear
precision for the planar ``top view mesh'' $\SS'$ (i.e., $\Delta_\phi f=0$
if $f$ is a linear function), and $w_{ij}\ge 0$ ensures semidefiniteness
and a maximum principle for $\Delta_\phi$\dash harmonic
functions. Equation \eqref{eq:deqz} can be written as $\Delta_\phi s = AF$.

\subsection{Surfaces in Isotropic Geometry} \label{sec:smooth}

It is worth wile to reconsider the basics of self\dash supporting surfaces
in the language of dual\dash isotropic geometry, which takes place in
$\R^3$ with the $z$ axis as a distinguished vertical direction. The basic
elements of this geometry are planes, having equation $z=f(x,y) = \alpha
x+\beta y+\gamma$. The gradient vector $\nabla f = (\alpha,\beta)$
determines the plane up to translation. A plane tangent to the graph of
the function $s(x,y)$ has gradient vector $\nabla s$.

There is the notion of {\em parallel points}:
	$
	(x,y,z) \parallel (x',y',z') \iff
	x=x',\ y=y'
	.$

In the differential geometry of surfaces one considers the {\em Gauss map}
$\sigma$ from a surface $S$ to a convex gauge body $\Phi$ by requiring
that corresponding points have parallel tangent planes.  Subsequently mean
curvature $H^\rel$ and Gaussian curvature $K^\rel$ {\em relative to
$\Phi$} are computed from the derivative $d\sigma$. Classically $\Phi$ is
the unit sphere (so that $\sigma$ maps each point its unit normal vector),
leading to the ordinary curvatures $H$ and $K$.


\paragraph{Computing Curvatures.}

In our setting, parallelity is a property of {\em points} rather than
lines, and the Gauss map $\sigma$ goes the other way, mapping the tangent
planes of the gauge body $z=\phi(x,y)$ to the corresponding tangent plane
of the surface $z=s(x,y)$. If we know which point a plane is attached to,
then it is determined by its gradient. So we simply write
	$$\nabla \phi\overset\sigma\longmapsto\nabla s.
	$$
 By moving along a curve $\uw(t)=(x(t),y(t))$ in the parameter domain we
get the first variation of tangent planes:
	$
	{d\over dt}\nabla \phi|_{\uw(t)} =
	(\Hess\phi)\dot\uw
	$.
 This yields the derivative
	$	
	(\Hess\phi)\dot\uw \overset{d\sigma}\longmapsto
	(\Hess s)\dot\uw $,
 for all $\dot\uw$, and the matrix of $d\sigma$ is found as
$(\Hess\phi)^{-1}(\Hess s)$.  By definition, curvatures of the surface $s$
{\em relative} to $\phi$ are found as
	\begin{align*}
		K_s^\rel
	& = \textstyle
		\det(d\sigma) =
		{\det\Hess s \over \det\Hess\phi} ,
	\\
		H_s^\rel
	&= \textstyle
		{1\over 2}\tr(d\sigma)
		= {1\over 2}\tr \left({M\over\det\Hess\phi} \Hess s\right)
		=  {\Delta_\phi s \over 2\det\Hess\phi}.
	\end{align*}
 The Maxwell paraboloid $\phi_0(x,y)={1\over 2}(x^2+y^2)$ is called the
{\em unit sphere} of isotropic geometry, its Hessian equals $E_2$.
Curvatures relative to that gauge body are not called ``relative''. We get
	\begin{align*}
	K_s = \det \Hess s,
		\
	H_s = {\Delta s \over 2},
		\
	K_s^\rel = {K_s\over K_\phi},
		\
	H_s^\rel =  {\Delta_\phi s \over 2 K_\phi}
			= {\Delta_\phi s\over \Delta_\phi \phi}
	\end{align*}
 (for the last formula we have used $\tr (M\Hess\phi)=\tr(E_2)=2$).

\paragraph{Relation to Self-supporting Surfaces.}

Applying the definitions above to the convex Airy stress potential $\phi$
of a self\dash supporting surface, we rewrite the balance conditions
\eqref{eq:conds} as
	\begin{equation}
	2 K_\phi H_s^\rel  = F.
	\label{equigeo}
	\end{equation}
 Let us draw some conclusions:

\begin{itemize}\itemsep-\parsep

\item Since $H^\rel_\phi=1$ we see that the load $F_\phi=2K_\phi$ is
admissible for the stress surface $\phi(x,y)$, which is hereby shown as
self\dash supporting. The quotient of admissible loads yields
	$
	 H_s^\rel = F/F_\phi.
	$

\item If the stress surface coincides with the Maxwell paraboloid, then
{\em constant loads characterize constant mean curvature surfaces},
because we get $K_\phi=1$ and $H_s=F/2$.

\item If $s_1,s_2$ have the same stress potential $\phi$, then
$H^\rel_{s_1-s_2}=H^\rel_{s_1}-H^\rel_{s_2}=0$, so $s_1-s_2$ is a
`relative' minimal surface.

\end{itemize}



\subsection{Meshes in Isotropic Geometry} \label{sec:discrete}

A general theory of curvatures of polyhedral surfaces with respect to a
polyhedral unit sphere was proposed by \cite{Pottmann2007b,bobenko-2010-ct},
and its dual complement
in isotropic geometry was elaborated by \cite{Pottmann2007}. As also
illustrated by Figure~\ref{fig:christoffel}, the mean curvature of a
self\dash supporting surface $\SS$ {\em relative} to its discrete Airy
stress potential is associated with the vertices of $\SS$. It is computed
from areas and mixed areas of faces in the polar polyhedra $\SS^*$ and
$\Phi^*$ which correspond to the vertex $\vw_i$:
	\begin{align*}
	H^\rel(\vw_i)
	&= {A_i(\SS,\Phi) \over A_i(\Phi,\Phi)},
	\quad\text{where}
	\\
		A_i (\SS,\Phi)
	&=
		\frac{1}{4}
		\sum_{k:f_k\in \text{1-ring}(\vw_i)}
		\det(\vw'^*_k, \ww'^*_{k+1})
		+ \det(\ww'^*_k, \vw'^*_{k+1}).
	\end{align*}
 The prime denotes the projection into the $xy$ plane, and summation is
over those dual vertices which are adjacent to $\vw_i$.
Replacing $\vw_k^*$ by $\ww_k^*$ yields
	$
		A_i (\Phi,\Phi)
	=
	\frac{1}{2}
		\sum
		\det(\ww'^*_k, \ww'^*_{k+1}).
	$

\begin{figure}[h]
 \centering
 \begin{overpic}[width=.8\columnwidth]{fig/christoffel}
	\put(0,8){$\SS$}
	\put(17,12){$\vw_i$}
	\put(0,30){$\Phi$}
	\cput(70,37){$\SS^*$}
	\cput(80,0){$\Phi^*$}
	\color{blau}
	\lput(52,18){$\ww_0^*$}
	\lput(65,22){$\vw_0^*$}
	\lput(60,0){$\ww_1^*$}
	\lput(58,32){$\vw_1^*$}
	\put(91,8){$\ww_2^*$}
	\put(82,37){$\vw_2^*$}
	\put(82,21){$\ww_3^*$}
	\put(89,26){$\vw_3^*$}
	\color{gelb}
	\cput(8,37){$f_0^\Phi$}
	\cput(8,18){$f_0^\SS$}
	\cput(13,27){$f_1^\Phi$}
	\cput(13,9){$f_1^\SS$}
	\cput(29,32){$f_2^\Phi$}
	\cput(29,12){$f_2^\SS$}
	\cput(22,42){$f_3^\Phi$}
	\cput(22,22){$f_3^\SS$}
 \end{overpic}
 \caption{Mean curvature of a vertex $\vw_i$ of $\SS$: Corresponding
edges of the polar duals $\SS^*$, $\Phi^*$ are parallel, and mean curvature
according to \protect\cite{Pottmann2007b} is computed from the
vertices polar to faces adjacent to $\vw_i$. For valence 4 vertices
the case of zero mean curvature shown here is characterized
by parallelity of non\dash corresponding diagonals of corresponding
quads in $\SS^*,\Phi^*$.}
 \label{fig:christoffel}
 \end{figure}

\proclaim Proposition.
 If $\Phi$ is the Airy surface of a thrust network $\SS$, then the mean
curvature of $\SS$ relative to $\Phi$ is computable as
	\begin{equation}
	\label{eq:Hrel}
		H^\rel(\vw_i)
	=
		{\sum_{j\sim i} w_{ij} (s_j-s_i)
		\over \sum_{j\sim i} w_{ij} (\phi_j-\phi_i) }
	=
		{\Delta_\phi s\over \Delta_\phi \phi}\Big|_{\vw_i}.
	\end{equation}

This is an immediate consequence of the following

\proclaim Lemma.
	$
	2A_i(\SS,\Phi)
	= \sum_{j\sim i} w_{ij} (s_j-s_i).
	$

\begin{proof} Consider edges $\ew'_1,\dots,\ew'_n$ emanating from
$\vw_i'$, and the dual cycles in $\Phi^{*\prime}$ and $\SS^{*\prime}$
which without loss of generality are given by vertices
$(\vw^{*\prime}_1,\dots,\vw^{*\prime}_n)$ and
$(\ww^{*\prime}_1,\dots,\ww^{*\prime}_n)$, respectively. The former has
edges $\ww'^*_{j+1}-\ww'^*_j = w_{ij} J\ew'_j$ (indices modulo $n$).

Without loss of generality $\vw_i=0$, so the vertex $\vw'^*_j$ equals the
gradient of the linear function $\xw\mapsto \<\vw'^*_j,\xw\>$ defined by
the properties $\ew'_{j-1}\mapsto s_{j-1}-s_i$, $\ew'_j\mapsto s_j-s_i$.
Corresponding edge vectors $\vw'^*_{j+1}-\vw'^*_j$ and
$\ww'^*_{j+1}-\ww'^*_j$ are parallel, because
$\<\vw'_{j+1}-\vw'_j,\ew'_j\>=(s_j-s_i)-(s_j-s_i)=0$. Now expand
$2A_i(\SS,\Phi)$:
	\begin{align*}
	& \mathrel{\hphantom{=}}
		{1\over 2}\sum
		\det(\ww'^*_j, \vw'_{j+1}) + \det(\vw'_j, \ww'^*_{j+1})
	\\
	&=
		{1\over 2}\sum
		\det(\ww'^*_j-\ww'^*_{j+1}, \vw'_{j+1})
		+ \det(\vw'_j, \ww'^*_{j+1}-\ww'^*_j)
		\\
	&=
		{1\over 2}\sum
		\det( - w_{ij} J\ew'_j, \vw'_{j+1})
	 	+ \det(\vw'_j,w_{ij} J\ew'_j)
	\\
	&= 	 \sum \det( \vw'_j, w_{ij} J\ew'_j)
	=	 \sum	w_{ij} \< \vw'_j, \ew'_j\>
	= 	 \sum  w_{ij} (s_j-s_i).
	\end{align*}
 Here we have used $\det(\aw,J\bw)=\<\aw,\bw\>$.
 \end{proof}


In order to discretize \eqref{equigeo}, we also need a discrete Gaussian
curvature, which is usually defined as a quotient of areas which
correspond under the Gauss mapping. We define
	$$
	K_\Phi(\vw_i) = {A_i(\Phi,\Phi) \over A_i},
	$$
 where $A_i$ is the Voronoi area of vertex $\vw_i'$ in the projected mesh
$\SS'$, which was used in \eqref{eq:deqz}.



 \begin{figure*}[tb]
	\centering
	%\includegraphics[width=0.24\textwidth]{fig/lilium.png} 
	%\includegraphics[width=0.24\textwidth]{fig/lilium-n.png}
	%\hfill
	%\includegraphics[width=0.24\textwidth]{fig/lilium-pillar-n.png} 
	%\includegraphics[width=0.24\textwidth]{fig/lilium-pq-n.png}
 \centerline{\begin{overpic}[width=.25\textwidth]{fig/lilium0.jpg}
		\cput(40.05,50){$\downarrow$}
		\cput(40,56){$|$}
		\cput(40,62){$|$}
		\cput(40,72){\small impossible feature}
		\put(0,0){(a)}
	\end{overpic}\hfill
	\begin{overpic}[width=.25\textwidth]{fig/lilium-n.jpg}
		\put(0,0){(b)}
		\color{gelb}
		\put(20,64){$\SS$}
	\end{overpic}\hfill
	\begin{overpic}[width=.25\textwidth]{fig/lilium-nstress.jpg}
		\put(0,0){(c)}
		\color{blau}
		\put(25,60){$\Phi$}
		\lput(45,0){$\SS'^*=\Phi'^*$}
	\end{overpic}\hfill
	\begin{overpic}[width=0.25\textwidth]{fig/lilium-pillar-n.jpg}
		\put(0,0){(d)}
	\end{overpic}}
 \caption{The top of the Lilium Tower (a) cannot stand as a masonry 
structure, because of a local bowel.
(b) Our algorithm finds a nearby self-supporting mesh without 
this impossible feature. (c) shows the corresponding Airy mesh $\Phi$ and 
reciprocal force diagram $\SS'^*$. (d) The user can edit the original 
surface, such as by specifying that the center of the surface is supported 
by a vertical pillar, and the self-supporting network adjusts accordingly}
 \label{fig:Lilium}
 \end{figure*}



\paragraph{Discrete Balance Equation.}

We now prove the discrete analogue to Equation \eqref{equigeo}.

\proclaim Theorem.
 A simply-connected mesh $\SS$ with vertices $\vw_i=(x_i,y_i,s_i)$
can be put into static equilibrium with vertical forces ``$A_iF_i$'' if
and only if there exists a combinatorially equivalent mesh $\Phi$ with
planar faces and vertices $(x_i,y_i,\phi_i)$, such that curvatures of
$\SS$ relative to $\Phi$ obey
	\begin{equation}
	2 K_\Phi(\vw_i) H^\rel(\vw_i) = F_i
	\label{eq:deqiso}
	\end{equation}
 at every interior vertex and every free boundary vertex $\vw_i$. $\SS$
can be put into compressive static equilibrium if and only if there exists
a convex such $\Phi$.

\begin{proof} The relation between equilibrium forces $w_{ij}\ew_{ij}$ in
$\SS$ and the polyhedral stress potential $\Phi$ has been discussed above,
and so has the equivalence ``$w_{ij}\ge 0$ $\iff$ $\Phi$ convex'' (see
e.g.\ \cite{Ash1988} for a survey of this and related results). It remains
to show that Equations \eqref{eq:deqtop} and \eqref{eq:deqiso} are
equivalent. This is the case because the proposition above implies
	$
	2 K(\vw_i) H^\rel(\vw_i) =
	2 \frac{A_i(\Phi,\Phi)}{A_i}
	\frac{A_i(\Phi,\SS)}{A_i(\Phi,\Phi)} =
	{1\over A_i}
	(\sum_{j\sim i} w_{ij} (s_j-s_i))
	= {1\over A_i} A_i F_i.
	$
	\end{proof}

\paragraph{Relation to discrete Laplace-Beltrami operators.}

For a given smooth surface $s(x,y)$ with Airy stress function $\phi$, and 
a given discrete top view $\SS'$, does there exist a polyhedral surface 
$\SS$ in equilibrium approximating $s(x,y)$? We restrict our attention to 
triangle meshes, where planarity of the faces of the discrete stress 
surface $\Phi$ is not an issue.
This question has several equivalent reformulations:

\begin{itemize}\itemsep-\parsep

\item Does $\SS'$ have a reciprocal diagram whose corresponding
Airy polyhedron $\Phi$ approximates the continuous Airy potential $\phi$?

\item Does $\SS'$ possess a ``perfect'' discrete Laplace\dash Beltrami operator 
$\Delta_\phi$ in the sense of Wardetzky et al.~\shortcite{wardetzky07}
whose weights are the edge length scalars of such a reciprocal diagram?

\end{itemize}

From \cite{wardetzky07} we know that perfect Laplacians exist only
on regular triangulations which are projections of convex polyhedra, so
the answer is affirmative after a possible re\dash triangulation. 
Previous sections suggest how to do this: If $\SS'$ lifts to a strictly
convex polyhedron $\Phi$ with vertices $(x_i,y_i,\phi(x_i,y_i))$,
then the polar dual
$\Phi^*$ projcts onto a reciprocal diagram with positive edge weights,
$\Delta_\phi$ has positive weights, and the vertices $(x_i,y_i,s_i)$ of $\SS$
are found by solving the discrete Poisson problem $(\Delta_\phi s)_i=A_iF_i$, 
which yields a mesh approximating $s(x,y)$.

We see that we may choose the {\em vertices} of the top view
$\SS'=\Phi'$ arbitrarily and lift them to vertices $(x_i,y_i,\phi(x_i,y_i))$
of $\Phi$. The edges are determined by finding a triangle mesh convex
hull of these vertices (if $\phi(x,y)={1\over 2}(x^2+y^2)$ this
constructs a Delaunay triangulation, and $\Delta_\phi$ is the cotan
Laplacian). We conclude: A smooth self-supporting surface can be
approximated by a discrete self\dash supporting triangular mesh
for any sampling of the surface.

\begin{figure*}[t]
	%\includegraphics[width=0.24\textwidth]{fig/build.png}
	%\includegraphics[width=0.24\textwidth]{fig/build-n.png}
	%\hfill
	%\includegraphics[width=0.24\textwidth]{fig/build-edited.png}
	%\includegraphics[width=0.24\textwidth]{fig/build-edited-n.png}
\centerline{
	\begin{overpic}[width=0.25\textwidth]{fig/build-white.jpg}
	\end{overpic}\hfill
	\begin{overpic}[width=0.25\textwidth]{fig/build-n.jpg}
	\end{overpic}\hfill
	\begin{overpic}[width=0.25\textwidth]{fig/build-edited-white.jpg}
	\end{overpic}\hfill
	\begin{overpic}[width=0.25\textwidth]{fig/build-edited-n.jpg}
	\end{overpic}}

\caption{The user-designed reference mesh (left) is not self-supporting, but our algorithm finds a nearby perturbation of the reference surface (middle-left) that is in equilibrium. As the user makes edits to the reference surface (middle-right), the thrust network automatically adjusts (right). \label{fig:vault}}
\end{figure*}

\section{Thrust Networks from Reference Meshes} \label{sec:opt} 

Consider
now the problem of taking a given reference mesh, $\RR$
and finding a combinatorically equivalent mesh $\SS$ in static equilibrium
approximating $\RR$. The loads on $\SS$ include user-prescribed loads as
well as the dead load caused by the mesh's own weight.
Conceptually, finding $\SS$ amounts to minimizing some
formulation of distance between $\RR$ and $\SS$, subject to constraints
\eqref{eq:deqtop}, \eqref{eq:deqz}, and $w_{ij} \geq 0$. For any choice of
distance this minimization will be a nonlinear, non-convex,
inequality-constrained variational problem that cannot be efficiently
solved in practice. Instead we propose a staggered optimization algorithm:

\begin{enumerate}\itemsep-\parsep\setcounter{enumi}{-1}

\item Start with an initial guess $\SS = \RR$.

\item \label{step2} Estimate the self\dash load on the vertices of $\SS$, using
their current positions.

\item \label{step3} Fixing $\SS$, fit an associated stress surface $\Phi$.

\item \label{step4} Alter positions $\vw_i$ to improve the fit.

\item Repeat from Step~\ref{step2} until convergence.

\end{enumerate}

\paragraph{Step~\ref{step2}: Estimating Self-load.}

The dead load due to the surface's own weight 
depends not only on the top view of $\SS$, but
also on the surface area of its faces. To avoid adding nonlinearity to the
algorithm, we estimate the load coefficients $F_i$ at the beginning of each
iteration, and assume they remain constant until the next iteration.
We estimate the load ``$A_iF_i$''
associated with each vertex by calculating its Voronoi area on
each of its incident faces, and then multiplying by a user-specified
surface density $\rho$.

\paragraph{Step~\ref{step3}: Fit a Stress Surface.}

In this step, we fix $\SS$ and try to
fit a stress surface $\Phi$ subordinate to the top view $\SS'$ of the
primal mesh. We do so by searching for dihedral angles between the faces of $\Phi$
which minimize, in the least-squares sense, the error in force equilibrium
\eqref{eq:deqiso} and local integrability of $\Phi$. Doing so is
equivalent to minimizing the squared residuals of equations
\eqref{eq:deqz} and \eqref{eq:deqtop}, respectively, with the positions
held fixed:
	\begin{align}
	\min_{w_{ij}}\
	&
	\sum_i
	\Big\| \Forcevector -
		\sum_{j\sim i} w_{ij} (\vw_j-\vw_i) \Big\|^2
		\nonumber
	\\&
	\textrm{s.t.}\ \
		0 \leq w_{ij} \leq w_{\max}, \label{eq:wbounds}
\end{align}
 where the outer sum is over the interior and free boundary vertices, and
$w_\textrm{max}$ is an optional maximum weight we are willing to assign
(to limit the amount of stress in the surface). This convex, sparse,
box-constrained least-squares problem \cite{BCLS}always has a solution. If the
objective is $0$ at this solution, the faces of $\Phi$ locally integrate
to a stress surface satisfying \eqref{eq:deqiso}, and so $\Phi$ certifies
that $\SS$ is self-supporting -- we are done. Otherwise, $\SS$ is not
self-supporting and its vertices must be moved.

\paragraph{Step~\ref{step4}: Alter Positions.} In the previous step we fit as best as
possible a stress surface $\Phi$ to $\SS$. There are two possible kinds of
error with this fit: the faces around a vertex (equivalently, the
reciprocal diagram) might not close up; and the resulting stress forces
might not be exactly in equilibrium with the loads. These errors can be
decreased by modifying the top view and heights of $\SS$, respectively. It
is possible to simply solve for new vertex positions that put $\SS$ in
static equilibrium, since equations \eqref{eq:deqtop} and \eqref{eq:deqz}
with $w_{ij}$ fixed form a square linear system that is typically
nonsingular.

While this approach would yield a self-supporting $\SS$, this mesh is
often far from the reference mesh $\RR$, since any local errors in the
stress surface from Step~\ref{step3} amplify into global errors in $\SS$. We propose
instead to look for new positions that decrease the imbalance in the
stresses and loads, while also penalizing drift away from the reference
mesh:
	\begin{align*}
	\min_{\vw}
	\ &
	\sum_i \Big\|
		\Forcevector -
		\sum_{j\sim i} w_{ij} (\vw_j - \vw_i)
		\Big\|^2
	\\ &
	+ \alpha \sum_i
		\<\nw_i, \vw_i - \vw^0_i \>^2
		+ \beta \big\|\vw - \vw^0_P\big\|^2,
\end{align*}
 where $\vw^0_i$ is the position of the $i$-th vertex at the start of this
step of the optimization, $\nw_i$ is the starting vertex normal (computed
as the average of the incident face normals), $\vw^0_P$ is the projection
of $\vw^0$ onto the reference mesh, and $\alpha > \beta$ are penalty
coefficients that are decreased every iteration of steps 2-4 of the
algorithm. The second term allows $\SS$ to slide over itself (if doing so
improves equilibrium) but penalizes drift in the normal direction. The
third term, weaker than the second, regularizes the optimization by
preventing large drift away from the reference surface or excessive
tangetial sliding.

Solving this weighted least-squares problem amounts to solving a sparse,
symmetric linear system. While the MINRES algorithm~\cite{paige75} is likely
the most robust algorithm for solving this system, in practice we have
observed that the method of conjugate gradients works well despite the
potential ill-conditioning of the objective matrix.

\paragraph{Limitations.}
This algorithm is not guaranteed to always converge; this fact is not surprising from the physics of the problem (if the boundary of the reference mesh encloses too large of a region, $w_{\max}$ is set too low, and the density of the surface too high, a thrust network in equilibrium simply does not exist -- the vault is too ambitious and cannot be built to stand; pillars are needed.) We can, however, make a few remarks. Step~\ref{step3} always decreases the equilibrium energy
	$$E=\sum_i \Big\| \Forcevector 
		- \sum_{j\sim i} w_{ij} (\vw_j - \vw_i)\Big\|^2$$
and Step~\ref{step4} does as well as $\beta \to 0$. Moreover, as $\alpha \to 0$ and $\beta \to 0$, Step~\ref{step4} approaches a linear system with as many equations as unknowns; if this system has full rank, its solution sets $E=0$. These facts suggest that the algorithm should generally converge to a thrust network in equilibrium, provided that Step~\ref{step2} does not increase the loads by too much at every iteration, and this is indeed what we observe in practice. One case where this assumption is guaranteed to hold is if the thickness of the surface is allowed to freely vary, so that it can be chosen so that the surface has uniform density over the top view.

If the linear system in Step~\ref{step4} is singular and infeasible, the algorithm can stall at $E > 0$. This failure occurs, for instance, when an interior vertex has height $z_i$ lower than all of its neighbors, and Step~\ref{step3} assigns all incident edges to that vertex a weight of zero: clearly no amount of moving the vertex or its neighbors can bring the vertex into equilibrium. We avoid such degenerate configurations by bounding weights slightly away from zero in \eqref{eq:wbounds}, trading increased robustness for slight smoothing of the resulting surface.




	\begin{figure*}[htb]
	\centering
	\includegraphics[height=0.15\textwidth]{fig/cas.jpg}
	\includegraphics[height=0.15\textwidth]{fig/cas-n.jpg}
	\hfill
\caption{A freeform surface (a) needs adjustments around the entrance arch and between the two pillars in order to be self-supporting;
our algorithm finds the nearby surface in equilibrium (b) that incorporates these changes. From this self-supporting surface and its
\todo{corresponding discrete stress surface, a PQ remeshing of the surface can be found by discretizing relative principal curvature
directions~\secref{sec:special}}. }
 \label{fig:cas} 
 \end{figure*}

\section{Design of Self-Supporting Surfaces} \label{sec:design}

The optimization algorithm described in the previous section forms the
basis of an interactive design tool for self-supporting surfaces. Users
manipulate a mesh representing a reference surface, and the computer
searches for a nearby thrust network in equilibrium (see e.g.\
Figure~\ref{fig:vault}). Fitting this thrust
network does not require that the user specify boundary tractions, and
although the top view of the reference mesh is used as an initial guess
for the top view of the thrust network, the search is not restricted to
this top view.

Some features of the design tool include:

\begin{itemize}\itemsep-\parsep

\item Handle-based 3D editing of the reference mesh using Laplacian
coordinates~\cite{Lipman2004,Sorkine2003} to extrude vaults, insert
pillars, and apply other deformations to the reference mesh. Handle-based
adjustments of the heights, keeping the top view fixed, and deformation of
the top view, keeping the heights fixed, are also supported. The thrust
network adjusts interactively to fit the deformed positions, giving the
usual visual feedback about the effects of her edits on whether or not the
surface can stand.

\item Specification of boundary conditions. Points of contact between the
reference surface and the ground or environment are specified by
``pinning'' vertices of the surface, specifying that the thrust network
must coincide with the reference mesh at this point, and relaxing the
condition that forces must be in equilibrium there.

\item Interactive adjustment of surface density $\rho$, external loads,
and maximum permissible stress per edge $w_{\textrm{max}}$, with visual
feedback of how these parameters affect the fitted thrust network.

\item Upsampling of the thrust network through Catmull\dash Clark
subdivision~\cite{catmull78} and polishing of the resulting refined thrust
network using optimization \secref{sec:opt}.

\item Visualization of the stress surface $\RR$ dual to the thrust network and corresponding reciprocal diagram.

\end{itemize}



\paragraph{Example: Vault with Pillars.} As an example of the design and optimization workflow, consider a rectangular vault with six pillars, free boundary conditions along one edge, fixed boundary conditions along the others, and a tower extruded from the top of the surface (see Figure \ref{fig:vault}). This surface is neither convex nor simply connected, and exhibits a mix of boundary conditions, none of which cause our algorithm any difficulty; it finds a self-supporting thrust network near the designed reference mesh. The user is now free to make edits to the reference mesh, and the thrust network adapts to these edits, providing the user feedback on whether these designs are physically realizable.


\paragraph{Example: Top of the Lilium Tower.}

Consider the top portion of the steel-glass exterior surface of the Lilium 
Tower, which is currently being built in Warszaw (see Figure \ref{fig:Lilium}).
This surface contains a local minimum 
in its interior and so cannot possibly be self-supporting; more generally, 
the entire central portion of the surface has positive Gaussian curvature 
and points downwards, an impossible feature in a self-supporting mesh. 
Given this surface as a reference mesh, our algorithm constructs a nearby 
thrust network in equilibrium without the impossible feature. The user can 
then explore how editing the reference mesh -- adding a pillar, for 
example -- affects the thrust network and its deviation from the reference 
surface.

\paragraph{Example: Freeform Structure with Two Pillars.}

The architect's experience and intuition has permitted him to design a freeform surface 
(see Figure \ref{fig:cas}) that is nearly self-supporting. Our algorithm reveals those
edits needed make the structure sound -- principally around the entrance arch, and the area
between the two pillars. From the thrust network, we also compute the unique conjugate curve
network that allows the surface to be remeshed using planar quads, using the algorithm
discussed in the next section.

\paragraph{Example: Swiss Cheese.}

  \begin{figure}[htb]
  \centerline{\includegraphics[width=0.5\columnwidth]{fig/cheese.jpg}\hfill
  \includegraphics[width=0.5\columnwidth]{fig/cheese-n.jpg}}
  \caption{This mesh with holes (left) requires large deformations
to both the top view and heights to render it self\dash supporting
(right).}
 	\label{fig:cheese2}
  \end{figure}

Cutting holes in a self-supporting surface interrupts force flow lines and causes dramatic global changes
to the surface stresses, often to the point that the surface is no longer in equilibrium. Whether a given surface
with many such holes can stand is far from obvious. Figures~\ref{fig:cheese2} 
show such an implausible and unstable 
surface; our optimization finds a nearby, equally implausible but stable surface without difficulty (see Figures~\ref{fig:cheese} and \ref{fig:cheese2}, right).


\begin{figure*}[t]
\centerline{\raise.02\textwidth
	\hbox{\begin{overpic}[width=.25\textwidth]{fig/curvature1}
		\put(55,5){$\phi(x,y)$}
		\put(55,51){$s(x,y)$}
		\put(0,-4){(a)}
	\end{overpic}}\hfill
	\begin{overpic}[width=0.25\textwidth]{fig/lilium-vf.jpg}
		\put(0,4){(b)}
	\end{overpic}\hfill
	\begin{overpic}[width=0.25\textwidth]{fig/lilium-pq.jpg}
		\put(0,4){(c)}
	\end{overpic}\hfill
	\begin{overpic}[width=0.25\textwidth]{fig/lilium-pq-n.jpg}
		\put(0,4){(d)}
	\end{overpic}}
	\vspace*{-0.01\textwidth}
\caption{Planar quad remeshing of self-supporting surfaces.
(a) A planar quad mesh approximating a self\dash supporting surface
$s(x,y)$ with stress potential $\phi(x,y)$ is guided by the principal
curvature directions of $s$ relative to $\phi$ (found from eigenvectors of
$(\protect\Hess\phi)^{-1}\protect\Hess s$). For the 
`Lilium tower' surface of Figure~\protect\ref{fig:Lilium}, the
principal directions (b) yield the planar quad remeshing (c) 
which is close to self\dash supporting;
subsequent small changes make it self\dash supporting (d).}
 \label{fig:lilium:pq} 
 \end{figure*}

\section{Special Self-Supporting Surfaces} \label{sec:special}

\paragraph{PQ Meshes.}

Meshes with {\em planar} faces are of particular interest in architecture, 
so in this section we discuss how to remesh a given thrust network in 
equilibrium such that it becomes a quad mesh with planar faces (again in 
equilibrium). For this purpose we first demonstrate how to find a quad 
mesh $\SS$ with vertices $\vw_{ij}=(x_{ij},y_{ij},s_{ij})$ which 
approximates a given continuous surface $s(x,y)$ equipped with an 
equilibrium stress potential $\phi(x,y)$.

It is known that $\SS$ must approximately follow a network of conjugate
curves in the surface (see e.g.\ \cite{Liu2006}). We can derive this
condition in an elementary way as follows: Using a Taylor expansion, we
compute the volume of the convex hull of the quadrilateral $\vw_{ij}$,
$\vw_{i+1,j}$, $\vw_{i+1,j+1}$, $\vw_{i,j+1}$, assuming the vertices lie
exactly on the surface $s(x,y)$. This results in
	\begin{align*}
	&\textstyle
	\text{vol} =
	{1\over 6}\det(\aw_1,\aw_2,(\aw_1)^T\,\Hess s\,\aw_2) + \cdots,
	\\
	\text{where}\
	& \textstyle
	\aw_1={x_{i+1,j}-x_{ij}\choose y_{i+1,j}-y_{ij}},\quad
	\aw_2={x_{i,j+1}-x_{ij}\choose y_{i,j+1}-y_{ij}},
	\end{align*}
 and the dots indicate higher order terms. We see that planarity requires
$(\aw_1)^T\,\Hess s\,\aw_2=0$. In addition to the mesh $\SS$ approximating the surface $s(x,y)$, the corresponding polyhedral
Airy surface $\Phi$ must approximate $\phi(x,y)$; thus we get the conditions
	$$
	(\aw_1)^T\,\Hess s\,\,\aw_2=
	(\aw_1)^T\,\Hess \phi\,\,\aw_2= 0.
	$$
 $\aw_1,\aw_2$ are therefore eigenvectors of $(\Hess\phi)^{-1}\Hess s$. In view
of \S\ref{sec:smooth}, $\aw_1,\aw_2$ indicate the principal directions of
the surface $s(x,y)$ relative to $\phi(x,y)$ (see
Figure~\ref{fig:lilium:pq}a).

In the discrete case, where $s,\phi$ are not given as continuous surfaces,
but are represented by a mesh in equilibrium and its Airy mesh, we use the
techniques of Schiftner~\shortcite{Schiftner2007} and Cohen\dash Steiner
and Morvan \shortcite{Cohen-Steiner2003} to approximate the Hessians
$\Hess s$, $\Hess\phi$, compute principal directions as eigenvectors of
$(\Hess\phi)^{-1}\Hess s$, and subsequently find meshes $\SS,\Phi$
approximating $s,\phi$ which follow those directions. Global
optimization now makes $\SS,\Phi$ a valid thrust network with discrete stress
potential. Convexity of $\Phi$ ensures that $\SS$ is self\dash supporting.

	\begin{figure}[htb]
  \centerline{\begin{overpic}[width=0.33\columnwidth]{fig/vault.jpg}
	\put(0,0)({(a)}
	\end{overpic}\hfill
  \begin{overpic}[width=0.33\columnwidth]{fig/vault-flat.jpg}
	\put(0,0)({(b)}
	\end{overpic}\hfill
  \begin{overpic}[width=0.33\columnwidth]{fig/vault-pq.jpg}
	\put(0,0)({(c)}
	\end{overpic}}
	\caption{Directly enforcing planarity of the faces of even a very simple self-supporting quad-mesh 
vault (a) results in a surface far removed from the original design (b). Starting instead from a
remeshing of the surface with edges following relative principal curvature directions yields a
self-supporting, PQ mesh far more faithful to the original (c).}
\label{fig:badpq}
\end{figure}

Note that the relative principal curvature directions give the \emph{unique}
curve network along which a planar quad discretization of a self-supporting surface
is possible. Taking an arbitrary non-planar quad mesh and attempting naive, simultaneous
enforcement of planarity and static equilibrium does not yield good results,
as shown in Figure~\ref{fig:badpq}. Figures~\ref{fig:lilium:pq}b--d 
and~\ref{fig:cas}c further illustrates the result of applying this 
procedure to self-supporting surfaces. 

\paragraph{Koenigs Meshes. UNFINISHED}

Consider a self\dash supporting thrust network $\SS$ and corresponding 
Airy mesh $\Phi$. Both $\SS$ and $\Phi$ are elements of the linear space 
of meshes which project onto $\SS'$. Any such mesh has vertices 
$\vw_i=(x_i,y_i,z_i)$, where the choice $z_i=s_i$ leads to $\SS$ and 
$z_i=\phi_i$ leads to $\Phi$. We already know that the vertical loads 
$A_iF_i$ which put $\SS$ into equilibrium are computed as $AF=\Delta_\phi 
s$.

We ask: Which perturbations $\SS+\RR$, having $z$ coordinates $z_i = s_i+r_i$, 
support the {\em same} vertical loads as $\SS$ does? Such a mesh must 
satisfy $\Delta_\phi(s+r)=\Delta_\phi s$, so $\Delta_\phi r = 0$.
This linear system is easily solved in principle.

\begin{figure*}
	\begin{overpic}[width=.5\textwidth]{fig/enneper.jpg}
		\color{gelb}
		\lput(12,0){$\Phi+\alpha\RR$}
		\lput(45,0){$\Phi$}
		\lput(77,0){$\Phi-\alpha\RR$}
	\end{overpic}\relax
	\begin{overpic}[width=.18\textwidth]{fig/enneper-minimal.jpg}
		\color{blau}
		\lput(35,0){$\RR$}
	\end{overpic}\hfill
 \begin{minipage}[b]{.28\textwidth}
 \caption{A `Koebe' mesh  $\Phi$ is self\dash supporting for unit dead
load. An entire family of self-supporting meshes with the same top view
is defined by $\SS_\alpha=\Phi+\alpha\RR$, where $\RR$ is chosen as $\Phi$'s 
Christoffel\dash dual.} \label{fig:enneper}
\end{minipage}
\end{figure*}

There is a nice explicit geometric construction of all such
harmonic functions in the case of quad meshes: Equation \eqref{eq:deqiso}
immediately leads to $H^\rel_\SS=H^\rel_{\SS+\RR}$, which is equivalent to
	$$
	H^\rel_\RR = 0.
	$$
 So $\RR$ is a {\em minimal surface}. Recall that \ $H^\rel_\RR$ is the
mean curvature of $\RR^*$ with respect to the Gauss image $\Phi^*$ in the
sense of \cite{Pottmann2007b}, where the star indicates the polar
polyhedron. We conclude that $\RR^*$ is constructed from $\Phi^*$ by the
condition of {\em parallel non\dash corresponding diagonals}, which is
also called Christoffel duality, and which can be seen in
Figure~\ref{fig:christoffel}. This condition determines $\RR^*$ uniquely
up to translation and scaling; thus $\RR$ is unique up to
scaling of $z$ coordinates and adding linear functions.

In general, however, $\RR$ does not exist, since only the so\dash
called {\em Koenigs meshes} possess a Christoffel dual.

\nix{ \begin{figure}[b]
  \begin{overpic}[width=.45\columnwidth]{fig/enneper}
	\end{overpic}\hfill
\begin{minipage}[b]{.48\columnwidth}
 \caption{From a self-supporting isotropic Koebe surface, it is possible
to construct a family of new self-supporting meshes, with identical top
view, using the Christoffel construction.} \label{fig:enneper}
\end{minipage}
 \end{figure}}

An  interesting special case occurs if $\Phi$ is an
\emph{isotropic Koebe mesh}, i.e., a PQ mesh whose edges
touch the Maxwell paraboloid. Since $\Phi$ approximates the Maxwell
paraboloid, we get $2K(\vw_i)H^{\rel}(\vw_i)$ $ \approx 1$ which means
$\Phi$ is self\dash supporting for unit load. Applying the Christoffel
dual construction described above yields a family a `minimal' mesh
$\RR$ and a family of meshes $\Phi+\alpha\RR$ which are self\dash 
supporting for unit load. Figure~\ref{fig:enneper} shows an example.

\section{Conclusion and Future Work}

\todo{TODO}
Some ideas:
\begin{itemize}
\item{Non-manifold surfaces}
\item{Non-vertical loads}
\item{Adaptive remeshing}
\end{itemize}

%\section*{Acknowledgements}

\bibliographystyle{acmsiggraph}


%\let\otb=\thebibliography
%\def\thebibliography#1{\otb{#1}\itemsep-5pt\footnotesize}
\bibliography{selfsupporting}




\end{document}

% Work into limitations?

\subsubsection{Possible pitfall}
Though the above optimization problem always has a solution, there is one unpleasant situation can arise: the optimum weights around \emph{all} edges adjacent to a vertex might be 0. I've observed this occur when the reference mesh has an interior local minimum at vertex $i$: obviously such a reference mesh cannot be a thrust network in equilibrium. If the neighbors of $i$ have few edges (such as in a quad mesh) the optimal weights for the edges around $i$ will be positive, since these edges are "needed" by $i$'s neighbors to put them in equilibrium. On the other hand, if $i$'s neighbors are edge-rich (such as in a triangle mesh) occasionally the best solution is to set all of $i$'s edges to have weight 0, effectively deleting vertex $i$ from the thrust network. A vertex whose surrounding edge weights are all zero (or near-zero) clearly can never satisfy \eqref{eq:dcond}, and giving the next step of the algorithm such a vertex as input causes numerical instabilities. Therefore in the current code I simply delete such vertices from the thrust network.

Reference meshes with large flat regions similarly cause numerical problems: again, in such regions no amount of altering the weights can improve the error in equation \eqref{eq:dcond}, leading the above optimization to sometimes assign many near-zero weights in such regions.


