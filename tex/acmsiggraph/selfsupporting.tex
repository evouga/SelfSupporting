%%% The ``\documentclass'' command has one parameter, based on the kind of
%%% document you are preparing.
%%%
%%% [annual] - Technical paper accepted for presentation at the ACM SIGGRAPH
%%%   or SIGGRAPH Asia annual conference.
%%% [sponsored] - Short or full-length technical paper accepted for
%%%   presentation at an event sponsored by ACM SIGGRAPH
%%%   (but not the annual conference Technical Papers program).
%%% [abstract] - A one-page abstract of your accepted content
%%%   (Technical Sketches, Posters, Emerging Technologies, etc.).
%%%   Content greater than one page in length should use the "[sponsored]"
%%%   parameter.
%%% [preprint] - A preprint version of your final content.
%%% [review] - A technical paper submitted for review. Includes line
%%%   numbers and anonymization of author and affiliation information.

\documentclass[review]{acmsiggraph}

\usepackage{amsmath}
\usepackage{amssymb}
\usepackage{amsthm}
\usepackage{par06}
\usepackage{overpic}
\usepackage{contour}\contourlength{1pt}
\usepackage{color}

\long\def\nix#1{\relax}


\def\div{\DID YOU RELLY WANT TO USE \div?}
\def\<{\mathchoice{\big\langle}{\langle}{\langle}{\langle}}
\def\>{\mathchoice{\big\rangle}{\rangle}{\rangle}{\rangle}}
%\def\lll{\mathopen{\mbox{$\<\hskip-.5ex\<$}}}
%\def\rrr{\mathclose{\mbox{$\>\hskip-.5ex\>$}}}
\def\wh{\widehat}
%\def\II{\mbox{I\hskip-0.1exI}}
\newtheorem{theorem}{Theorem}
\newtheorem{prop}[theorem]{Proposition}
\def\Div{\mathop{{\rm div}}\nolimits}
\def\tr{\mathop{{\rm tr}}\nolimits}
\def\rel{{\mathord{{\rm rel}}}}
\def\const{{\mathord{\textrm{const}}}}
\def\ess{s}
\def\Hess#1{{\def\testess{#1}\nabla^2\ifx\testess\ess\!s\else #1\fi}}
\def\Hess#1{\text{$\nabla^2\hskip-.2ex #1$}}
\def\Forcevector{\Big(\mbox{\scriptsize
	\def\arraystretch{0.8}\begin{tabular}{@{\,}c@{\,}}
	0 \\ 0 \\ $A_i F_i$
	\end{tabular}}\Big)}
\hyphenation{pa-rab-ol-loid War-detz-ky}

\def\lput(#1,#2)#3{\put(#1,#2){\hbox to 0pt{\hss{#3}}}}
\def\cput(#1,#2)#3{\put(#1,#2){\hbox to 0pt{\hss{#3}\hss}}}
\definecolor{blau}{rgb}{0.15,0.2,0.3}
%\definecolor{rot}{rgb}{0.7,0.5,0.2}
\definecolor{drot}{rgb}{0.7,0,0.1}
\definecolor{grey}{rgb}{0.6,0.6,0.6}
\definecolor{lightgrey}{rgb}{0.8,0.8,0.8}
\definecolor{gelb}{rgb}{.55,.40,.1}


\outer\def\proclaim #1. #2\par{\noindent{\bf#1.\enspace}{\it#2\par}}


%\usepackage{mathrsfs}
\def\SS{{\mathcal S}}
\def\RR{{\mathcal R}}


\newcommand{\todo}[1]{\textcolor{red}{#1}}
\newcommand{\secref}[1]{(\S\ref{#1})}

%%% If you are submitting your paper to one of our annual conferences - the
%%% ACM SIGGRAPH conference held in North America, or the SIGGRAPH Asia
%%% conference held in Southeast Asia - there are several commands you should
%%% consider using in the preparation of your document.

%%% 1. ``\TOGonlineID''
%%% When you submit your paper for review, please use the ``\TOGonlineID''
%%% command to include the online ID value assigned to your paper by the
%%% submission management system. Replace '45678' with the value you were
%%% assigned.

\TOGonlineid{0043}

%%% 2. ``\TOGvolume'' and ``\TOGnumber''
%%% If you are preparing a preprint of your accepted paper, and your paper
%%% will be published in an issue of the ACM ``Transactions on Graphics''
%%% journal, replace the ``0'' values in the commands below with the correct
%%% volume and number values for that issue - you'll get them before your
%%% final paper is due.

\TOGvolume{0}
\TOGnumber{0}

%%% 3. ``TOGarticleDOI''
%%% The ``TOGarticleDOI'' command accepts the DOI information provided to you
%%% during production, and which makes up the URLs which identifies the ACM
%%% article page and direct PDF link in the ACM Digital Library.
%%% Replace ``1111111.2222222'' with the values you are given.

\TOGarticleDOI{1111111.2222222}

%%% 4. ``\TOGprojectURL'', ``\TOGvideoURL'', ``\TOGdataURL'', ``\TOGcodeURL''
%%% If you would like to include links to personal repositories for auxiliary
%%% material related your research contribution, you may use one or more of
%%% these commands to define an appropriate URL. The ``\TOGlinkslist'' command
%%% found just before the first section of your document will add hyperlinked
%%% icons to your document, in addition to hyperlinked icons which point to
%%% the ACM Digital Library article page and the ACM Digital Library-held PDF.

\TOGprojectURL{}
\TOGvideoURL{}
\TOGdataURL{}
\TOGcodeURL{}

%%% Replace ``PAPER TEMPLATE TITLE'' with the title of your paper or abstract.

\title{Design of Self-supporting Surfaces}

%%% The ``\author{}'' command takes the names and affiliations of each of the
%%% authors of your paper or abstract. The ``\thanks{}'' command takes the
%%% contact information for each author.
%%% For multiple authors, separate each author's information by the ``\and''
%%% command.

\author{
	Etienne Vouga
	\\ Columbia Univ. / KAUST
\and
	Mathias H\"obinger
	\\ Evolute / TU Wien
\and
	Johannes Wallner
	\\ TU Graz / TU Wien
\and 
	Helmut Pottmann
	\\ KAUST
}

%%% The ``pdfauthor'' command accepts the authors of the work,
%%% comma-delimited, and adds this information to the PDF metadata.

\pdfauthor{Anonymous}

%%% Keywords that describe your work. The ``\keywordlist'' command will print
%%% them out.

\keywords{Discrete differential geometry, architectural geometry,
self\dash supporting masonry, thrust networks,
reciprocal force diagrams, discrete Laplace operators, 
isotropic geometry, mean curvature}

%%% The ``\begin{document}'' command is the start of the document.

%%% If you have user-defined macros, you may include them here.

% example of a user-defined macro called ``remark.''
% \newcommand{\remark}[1]{\textcolor{red}{#1}}

\begin{document}

%%% A ``teaser'' image appears under the title and affiliation information,
%%% horizontally centered, and above the two columns of text. This is OPTIONAL.
%%% If you choose to have a ``teaser'' image, it needs to be placed between
%%% ``\begin{document}'' and ``\maketitle.''

%\teaser{
%   \includegraphics[height=1.5in]{images/sampleteaser}
%   \caption{Spring Training 2009, Peoria, AZ.}
%}

%%% The ``\maketitle'' command must appear after ``\begin{document}'' and,
%%% if you have one, after the definition of your ``teaser'' image, and
%%% before the first ``\section'' command.

\maketitle

%%% Your paper's abstract goes in its own section.

\begin{abstract} Self\dash supporting masonry is one of the most ancient 
and elegant techniques for building curved shapes. Because of the
very geometric nature of their failure, analyzing and modeling such strutures
is more a geometry processing problem than 
one of classical continuum mechanics. In this paper we use the thrust network method 
of analysis and present an iterative nonlinear optimization algorithm for 
efficiently approximating freeform shapes by self\dash supporting ones. 
The rich geometry of thrust networks leads us to
close connections between different topics of discrete differential geometry,
such as a finite\dash element discretization of the Airy stress potential, perfect graph 
Laplacians, and the problem of computing admissible loads via curvatures of polyhedral 
surfaces. This geometric viewpoint allows us, in particular, to remesh self\dash supporting 
shapes by self\dash supporting quad meshes with planar faces.

\end{abstract}

%%% ACM Computing Review (CR) categories.
%%% See <http://www.acm.org/class/1998/> for details.
%%% The ``\CRcat'' command takes four arguments.

\begin{CRcatlist}
  \CRcat{I.3.5}{Computer Graphics}{Computational Geometry and Object Modeling}{Curve, surface, solid, and object representations};
\end{CRcatlist}

%%% The ``\keywordlist'' command prints out the keywords.

\keywordlist

%%% The ``\TOGlinkslist'' command will insert hyperlinked icon(s) to your
%%% paper. This includes, at a minimum, hyperlinked icons to the ACM article
%%% page and the ACM Digital Library-held PDF. If you added URLs to
%%% ``\TOGprojectURL'' or the other, similar commands, they will be added to
%%% the list of icons.
%%% Note: this functionality only works for annual-conference papers.

\TOGlinkslist

%%% The ``\copyrightspace'' command
%%% Do not remove this command.

\copyrightspace

%%% This is the first section of the body of your paper.

%\insert\footins{\vspace*{4cm}}

\section{Introduction}


Vaulted masonry structures are among the simplest and at the same time 
most elegant solutions for creating curved shapes in building 
construction. For this reason they have been an object of interest 
since antiquity; large, non\dash convex examples of such structures include gothic 
cathedrals. They continue to be an active topic of research in today's 
engineering community.


Our paper is concerned with a combined geometry+statics analysis of {\em 
self\dash supporting} masonry and with tools for the interactive modeling 
of freeform self\dash supporting structures. Here ``self\dash supporting'' 
means that the structure, considered as an arrangement of blocks (bricks, 
stones), holds together by itself, with additional support present only during 
construction. Our analysis is based on the following assumptions, which follow the classic 
\cite{Heyman66}:


{\it Assumption 1:} Masonry has no tensile strength, but the individual 
building blocks do not slip against each other (because of friction or 
mortar). On the other hand, their compressive strength is sufficiently 
high so that failure of the structure is by a sudden change in geometry
and not by material failure.

{\it Assumption 2 (The Safe Theorem)}: If a system of forces can be found 
which is in equilibrium with the load on the structure and which is 
contained within the masonry envelope then the structure will carry the 
loads, although the actual forces present may not be those postulated.

Our approach is twofold: We first give an overview of the continuous case 
of a smooth surface under stress, which turns out to be governed locally by the 
so\dash called Airy stress function. This mathematical 
model is called a membrane in the engineering literature and has been 
applied to the analysis of masonry before. The surface is self\dash 
supporting if and only if stresses are entirely compressive (i.e., the 
Airy function is convex). For computational purposes, stresses are 
discretized as a fictitious {\em thrust network} \cite{Block07} contained 
in the masonry structure; this network is a system of forces in equilibrium with
the structure's deadload. It can be interpreted as a 
finite element discretization of the continuous case, and it turns out to 
have very interesting geometry, with the Airy stress function becoming a
polyhedral surface directly related to a reciprocal force diagram. 

While previous work in architectural geometry was mostly concerned
with aspects of rationalization and purely geometric side\dash conditions
which occur in freeform architecture, the focus of this paper is design with
{\em statics} constraints. In particular, our 
contributions are the following:


	\begin{figure}[t] 
	\includegraphics[width=\columnwidth]{arch-fig/cheesevault78.jpg} 
	\caption{Surfaces with irregularly placed holes almost 
never stand by themselves when built from bricks; for those that do,
stability is not obvious by inspection. The surface shown is produced by 
finding the nearest self\dash supporting shape from a given freeform
geometry. The image also illustrates
the fictitious thrust network used in our algorithm,
with edges' cross-section and coloring visualizing the magnitude of
forces (warmer colors represent higher stresses.)}
	\label{fig:cheese}
\end{figure}




\paragraph{Contributions.} 

\begin{list}{$\bullet$}{\leftmargin0pt\itemindent1em}

\item We connect the physics of self\dash supporting surfaces with 
vertical loads to the geometry of isotropic 3\dash space, with the 
direction of gravity as the distinguished direction \secref{sec:smooth}. 
Taking the convex Airy potential as unit sphere, one can express the 
equations governing self\dash supporting surfaces in terms of curvatures.


\item We consider the known constructions of polyhedral thrust networks and 
their reciprocal diagrams, and give an 
interpretation of the equilibrium conditions in terms of discrete 
curvatures \secref{sec:discrete}.

\item The graph Laplacian derived from a thrust network with compressive 
forces is a ``perfect'' one \secref{sec:thrustnetworks}.
We show how it appears in the analysis and 
establish a connection with mean curvatures which are otherwise defined 
for polyhedral surfaces.


\item We present an optimization algorithm for efficiently finding a 
thrust network near a given arbitrary reference surface \secref{sec:opt}, 
and build a tool for interactive design of self\dash supporting surfaces 
based on this algorithm \secref{sec:design}.

\item We exploit the geometric relationships between a self\dash 
supporting surface and its stress potential in order to find particularly 
nice families of self\dash supporting surfaces, especially planar 
quadrilateral representations of thrust networks \secref{sec:special}.

\item We demonstrate the versatility and applicability of our approach to 
the design and analysis of large\dash scale masonry and steel\dash glass 
structures.

\end{list}


\paragraph{Related Work.}

Unsupported masonry has been an active topic of research in the 
engineering community. The foundations for the modern approach were laid 
by Jacques Heyman \shortcite{Heyman66} and are available as the textbook 
\cite{Heyman95}. The theory of reciprocal force diagrams in the planar
case was studied by Maxwell \cite{Maxwell64};
a unifying view on polyhedral surfaces, compressive 
forces and corresponding ``convex'' force diagrams is presented by 
\cite{Ash1988}. F.~Fraternali \shortcite{Fraternali2002a}, 
\shortcite{Fraternali2010} established a connection between the continuous 
theory of stresses in membranes and the discrete theory of forces in 
thrust networks, by interpreting the latter as a certain non-conforming 
finite element discretization of the former.

Several authors have studied the problem of finding discrete compressive 
force networks contained within the boundary of masonry structures; previous
work in this area includes \cite{O'Dwyer98} and
\cite{andreu-2007}. Fraternali~\shortcite{Fraternali2010} proposed solving 
for the structure's discrete stress surface, and examining its convex hull 
to study the structure's stability and susceptibility to cracking. 
Philippe Block's seminal thesis introduced {\it Thrust 
Network Analysis}, which pioneered the use of thrust networks and their
reciprocal diagrams for efficient and practical design of self\dash supporting
masonry structures. By first seeking a reciprocal diagram of the top view, guaranteeing equilibrium 
of horizontal forces, then solving for the heights that balance the 
vertical loads, Thrust Network Analysis linearizes the form-finding problem.
For a thorough overview of this methodology, see e.g.\ \cite{Block07,block09}. 
Recent work by Block and coauthors extends this method in the case where the reciprocal diagram 
is not unique; for different choices of reciprocal diagram, the optimal 
heights can be found using the method of least squares~\cite{vanmele2011}, 
and the search for the best such reciprocal diagram can be automated using 
a genetic algorithm~\cite{Block2011}.

Other approaches to the interactive design of self-supporting structures 
include modeling these structures as damped particle-spring 
systems~\cite{Kilian2005,barnes09}, and mirroring the rich tradition in 
architecture of designing self-supporting surfaces using hanging chain 
models~\cite{Heyman98}. Alternatively, masonry structures can be 
represented by networks of rigid blocks~\cite{Livesley92}, whose conditions 
on the structural feasibility were incorporated into procedural modeling 
of buildings~\cite{Whiting09}.

Algorithmic and mathematical methods relevant to this paper are work on 
the geometry of quad meshes with planar faces \cite{Glymph2004,Liu2006}, 
discrete curvatures for such meshes \cite{Pottmann2007b,bobenko-2010-ct}, 
in particular curvatures in isotropic geometry \cite{Pottmann2007}. 
Schiftner and Balzer \shortcite{Schiftner2010} discuss approximating a 
reference surface by a quad mesh with planar faces, whose layout is guided 
by statics properties of that surface.

%\cite{Koenderink2002}, etc PQ meshes

\section{Self-supporting Surfaces}

\subsection{The Continuous Theory}

In this paper we model masonry as a surface given by a height field $s(x,y)$ 
defined in some planar domain $\Omega$. We assume that there are vertical 
loads $F(x,y)$ --- usually $F$ represents the structure's own weight. By 
definition this surface is self\dash supporting if and only if there 
exists a field of compressive stresses which are in equilibrium with the 
acting forces. This is equivalent to existence of a field $M(x,y)$ of 
$2\times 2$ symmetric positive semidefinite matrices satisfying
	\begin{align}
	\Div (M\nabla s) = F, \quad
	\Div M &= 0,
	  \label{eq:conds}
	\end{align}
 where the divergence operator $\Div{u(x,y)\choose v(x,y}= u_x + v_y$ is 
understood to act on the columns of a matrix (see e.g.\ 
\cite{Fraternali2010}, \cite{Giaquinta1985}).

The condition $\Div M=0$ says that $M$ is locally the Hessian of a 
real\dash valued function $\phi$ (the {\em Airy stress potential}): With 
the notation
	\begin{align*}
	M =
	{\textstyle {m_{11} \ m_{12} \choose m_{12} \ m_{22}}}
	\iff	
	\wh M =
	{\textstyle {\hphantom{-}m_{22} \ -m_{12} \choose -m_{12}
		 \ \hphantom{-}m_{11}}}
	\end{align*}
 it is clear that $\Div M=0$ is an integrability condition for $\wh M$, so
locally there is a potential $\phi$ with
	\begin{align*}
	\wh M = \Hess\phi, \quad \text{i.e.,}\quad
	M = \wh{\Hess\phi}.
	\end{align*}
 If the domain $\Omega$ is simply connected, this relation holds globally. 
Positive semidefiniteness of $M$ (or equivalently of $\wh M$) 
characterizes {\em convexity} of the Airy potential $\phi$. The Airy 
function enters computations only by way of its derivatives, so global 
existence is not an issue.

{\it Remark:} Stresses at boundary points depend on the way the surface is 
anchored: A fixed anchor means no condition, but a free boundary with 
outer normal vector $\nw$ means $\<M \nabla s, \nw \> = 0$.


\paragraph{Stress Laplacian.} Note that $\Div M =0$ yields $\Div(M\nabla 
s)$ $ =$ $ \tr(M\Hess s)$, which we like to call $\Delta_\phi s$. The 
operator $\Delta_\phi$ is symmetric. It is elliptic (as a Laplace operator 
should be) if and only if $M$ is positive definite, i.e., $\phi$ is 
strictly convex. The balance condition \eqref{eq:conds} may be written as
	$
	\Delta_\phi s = F.
	$


\subsection{Discrete Theory: Thrust Networks}
\label{sec:thrustnetworks}

We discretize a self-supporting surface by a mesh $\SS=(V,E,F)$
(see Figure~\ref{fig:reciprocal}). Loads are again vertical, 
and we discretize them as force densities $F_i$ associated with vertices 
$\vw_i$. The load acting on this vertex is then given by $F_iA_i$, where 
$A_i$ is an area of influence (using a prime to indicate projection onto 
the $xy$ plane, $A_i$ is the area of the Voronoi cell of $\vw_i'$ w.r.t.\ 
$V'$). We assume that stresses are carried by the edges of the mesh: the 
force exerted on the vertex $\vw_i$ by the edge connecting $\vw_i,\vw_j$ 
is given by
	\begin{align*}
	w_{ij} (\vw_j-\vw_i),
	\quad
	\text{where}\quad
	w_{ij}=w_{ji}\ge 0.
	\end{align*}
 The nonnegativity of the individual weights $w_{ij}$ expresses the 
compressive nature of forces. The balance conditions at vertices then read 
as follows: With $\vw_i=(x_i,y_i,s_i)$ we have
	\begin{align}
	\sum\nolimits_{j\sim i}
		w_{ij} (x_j - x_i)
	=
	\sum\nolimits_{j\sim i}
		w_{ij} (y_j - y_i) &= 0,
			 \label{eq:deqtop} \\
	\sum\nolimits_{j\sim i}
		w_{ij} (s_j - s_i)
		&= A_i F_i.
			\label{eq:deqz}
	\end{align}
 A mesh equipped with edge weights in this way is a discrete \emph{thrust
network}. Invoking the safe theorem, we can state that a masonry structure
is self\dash supporting, if we can find a thrust network with compressive
forces which is entirely contained within the structure.

  \begin{figure}[t]
  \centering
  \begin{overpic}[width=.94\columnwidth]{fig/reciprocal}
	\put(0,33){$\SS$}
	\lput(13,33){$\vw_i$}
	\cput(14,25){\contour{white}{$A_iF_i$}}
	\color{gelb}
	\lput(100,19){$\SS'^*$}
	\color{blau}
	\put(0,9){$\SS'$}
	\color{drot}
	\put(1,0){$w_{ij} \ew_{ij}'$}
	\lput(63,3){$\ew_{ij}^*$}
  \end{overpic}\nix{
 \begin{overpic}[width=\columnwidth]{fig/reciprocal1}
        \color{gelb}
        \lput(100,15){$\SS'^*$}
        \color{blau}
        \put(1,25){$\SS$}
        \lput(7,1){$\SS'$}
        \lput(21,35){$\vw_i$}
        \cput(21,29){$F_i$}
        \color{drot}
        \lput(13,8){\contour{white}{$w_{ij} \ew_{ij}'$}}
        \lput(66,11){\contour{white}{$\ew_{ij}^*$}}
  \end{overpic}}\relax
 \caption{A thrust network $\SS$ with dangling edges indicating external 
forces (left). This network together with compressive forces which balance 
vertical loads $A_iF_i$ projects onto a planar mesh $\SS'$ with 
equilibrium compressive forces $w_{ij}\ew_{ij}'$ in its edges. Rotating 
forces by 90$^\circ$ leads to the reciprocal force diagram $\SS'^*$ 
(right).}
  \label{fig:reciprocal}
  \end{figure}

\paragraph{Reciprocal Diagram.}

Equations \eqref{eq:deqtop} have a geometric interpretation: with edge
vectors
	\begin{align*}
	\ew'_{ij} = \vw_j'-\vw_i'=(x_j, y_j) - (x_i, y_i),
	\end{align*}
 Equation \eqref{eq:deqtop} asserts that vectors $w_{ij} \ew_{ij}'$ form a
closed cycle. Rotating them by 90 degrees, we see that likewise
	\begin{align*}
	\ew_{ij}^{\prime *} = w_{ij} J \ew_{ij}', \quad \text{with}\quad
	J={\textstyle{0 \ -1 \choose 1 \ \hphantom{-}0}},
	\end{align*}
 form a closed cycle (see Figure \ref{fig:reciprocal}). If the mesh $\SS$ 
is simply connected, there exists an entire {\em reciprocal diagram} 
$\SS^{\prime *}$ which is a combinatorial dual of $\SS$, and which has 
edge vectors $\ew_{ij}'^*$. Its vertices are denoted by $\vw_i^{\prime 
*}$.

{\it Remark:} If $\SS'$ is a Delaunay triangulation, then the 
corresponding Voronoi diagram is an example of a reciprocal diagram.

\paragraph{Polyhedral Stress Potential.}

We can go further and construct a convex polyhedral ``Airy stress 
potential'' surface $\Phi$ with vertices $\ww_i=(x_i,y_i,\phi_i)$ 
combinatorially equivalent to $\SS$ by requiring that a primal face of 
$\Phi$ lies in the plane $z=\alpha x + \beta y + \gamma$ if and only if 
$(\alpha,\beta)$ is the corresponding dual vertex of $\SS'^*$ (see 
Figure~\ref{fig:polarity}). Obviously this condition determines $\Phi$ up 
to vertical translation. For existence see \cite{Ash1988}. The inverse 
procedure constructs a reciprocal diagram from $\Phi$. This procedure 
works also if forces are not compressive: we can construct an Airy mesh
$\Phi$ which has planar faces, but it will no longer be a convex 
polyhedron.

The vertices of $\Phi$ can be interpolated by a piecewise\dash linear 
function $\phi(x,y)$. It is easy to see that the derivative of $\phi(x,y)$ 
jumps by the amount $\|\ew_{ij}'^*\| = w_{ij}\|\ew_{ij}'\|$ when crossing 
over the edge $\ew'_{ij}$ at right angle, with unit speed. This identifies 
$\Phi$ as the Airy polyhedron introduced by \cite{Fraternali2002a} as a 
finite element discretization of the continuous Airy function (see also 
\cite{Fraternali2010}).

If the mesh is not simply connected, the reciprocal diagram and the Airy 
polyhedron exist only locally. Our computations do not require global existence.

  \begin{figure}[t]
 \centerline{\vphantom{\includegraphics[width=.94\columnwidth]
		{fig/reciprocal}}\relax
  \begin{overpic}[width=.94\columnwidth]{fig/beide}
	\lput(49,33){$\ww_k^*$}
	\lput(49,15){$\vw_k^{*\prime}$}
	\color{gelb}
	\put(86,10){$\SS'^*=\Phi^{*\prime}$}
	\put(83,30){$\Phi^*$}
	\color{blau}
	\cput(24,35){$\Phi$}
	\put(0,0){$\Phi'=\SS'$}
	\cput(4,31){$f_k$}
  \end{overpic}}
%  \begin{overpic}[width=.94\columnwidth]{fig/beide1}
%        \put(46,30){$\ww_k^*$}
%        \put(46,16){$\vw_k^{*\prime}$}
%        \color{gelb}
%        \put(84,10){$\Sw'^*=\Phi^{*\prime}$}
%        \put(78,30){$\Phi^*$}
%        \color{blau}
%        \cput(30,30){$\Phi$}
%        \lput(8,1){$\Phi'=\SS'$}
%        \cput(4,33){$f_k$}
%  \end{overpic}
 \caption{Airy stress potential $\Phi$ and its polar dual $\Phi^*$. $\Phi$ 
projects onto the same planar mesh as $\SS$ does, while $\Phi^*$ projects 
onto the reciprocal force diagram.  A primal face $f_k$ lies in the plane 
$z=\alpha x + \beta y + \gamma$ $\iff$ the corresponding dual vertex is 
$\ww_k^*=(\alpha,\beta,-\gamma)$.}
  \label{fig:polarity}
  \end{figure}


\paragraph{Polarity.}

Polarity with respect to the {\em Maxwell paraboloid} $z={1\over 2} 
(x^2+y^2)$ maps the plane $z=\alpha x + \beta y + \gamma$ to the point 
$(\alpha,\beta,-\gamma)$. Thus, applying polarity to $\Phi$ and projecting 
the result $\Phi^*$ into the $xy$ plane reconstructs the reciprocal 
diagram $\Phi^{*\prime}=\SS^{\prime *}$ (see Fig.~\ref{fig:polarity}).

\paragraph{Discrete Stress Laplacian.}

The weights $w_{ij}$ may be used to define a graph Laplacian $\Delta_\phi$ 
which on vertex\dash based functions acts as
	\begin{align*}
	\Delta_{\phi} s(\vw_i)=\sum\nolimits_{j\sim i} w_{ij}(s_j-s_i).
	\end{align*}
 This operator is a perfect discrete Laplacian in the sense of 
\cite{wardetzky07}, since it is symmetric by construction, Equation 
\eqref{eq:deqtop} implies linear precision for the planar ``top view 
mesh'' $\SS'$ (i.e., $\Delta_\phi f=0$ if $f$ is a linear function), and 
$w_{ij}\ge 0$ ensures semidefiniteness and a maximum principle for 
$\Delta_\phi$\dash harmonic functions. Equation \eqref{eq:deqz} can be 
written as $\Delta_\phi s = AF$.

Note that $\Delta_\phi$ is well defined even when the underlying meshes 
are not simply connected.

\subsection{Surfaces in Isotropic Geometry} \label{sec:smooth}

It is worthwhile to reconsider the basics of self\dash supporting surfaces 
in the language of dual\dash isotropic geometry, which takes place in 
$\R^3$ with the $z$ axis as a distinguished vertical direction. The basic 
elements of this geometry are planes, having equation $z=f(x,y) = \alpha 
x+\beta y+\gamma$. The gradient vector $\nabla f = (\alpha,\beta)$ 
determines the plane up to translation. A plane tangent to the graph of 
the function $s(x,y)$ has gradient vector $\nabla s$.

There is the notion of {\em parallel points}:
	$
	(x,y,z) \parallel (x',y',z') \iff
	x=x',\ y=y'
	.$

{\it Remark:} The Maxwell paraboloid is considered the unit sphere of isotropic 
geometry, and the geometric quantities considered above are assigned
specific meanings: The forces $\|\ew_{ij}^*\|=w_{ij}\|\ew_{ij}\|$
are dihedral angles of the Airy polyhedron $\Phi$, and also ``lengths'' of
edges of $\Phi^*$. We do not use this terminology in the sequel.

\paragraph{Curvatures.}

Generally speaking, in the differential geometry of surfaces one considers 
the {\em Gauss map} $\sigma$ from a surface $S$ to a convex unit sphere 
$\Phi$ by requiring that corresponding points have parallel tangent 
planes.  Subsequently mean curvature $H^\rel$ and Gaussian curvature 
$K^\rel$ {\em relative to $\Phi$} are computed from the derivative 
$d\sigma$. Classically $\Phi$ is the ordinary unit sphere $x^2+y^2+z^2=1$, 
so that $\sigma$ maps each point to its unit normal vector.



In our setting, parallelity is a property of {\em points} rather than 
planes, and the Gauss map $\sigma$ goes the other way, mapping the tangent 
planes of the unit sphere $z=\phi(x,y)$ to the corresponding tangent plane 
of the surface $z=s(x,y)$. If we know which point a plane is attached to, 
then it is determined by its gradient. So we simply write
	\begin{align*}
	\nabla \phi\overset\sigma\longmapsto\nabla s.
	\end{align*}
 By moving along a curve $\uw(t)=(x(t),y(t))$ in the parameter domain we
get the first variation of tangent planes:
	$
	{d\over dt}\nabla \phi|_{\uw(t)} =
	(\Hess\phi)\dot\uw
	$.
 This yields the derivative
	$	
	(\Hess\phi)\dot\uw \overset{d\sigma}\longmapsto
	(\Hess s)\dot\uw $,
 for all $\dot\uw$, and the matrix of $d\sigma$ is found as 
$(\Hess\phi)^{-1}(\Hess s)$.  By definition, curvatures of the surface $s$ 
{\em relative} to $\phi$ are found as
	\begin{align*}
		K_s^\rel
	& = \textstyle
		\det(d\sigma) =
		{\det\Hess s \over \det\Hess\phi} ,
	\\
		H_s^\rel
	&= \textstyle
		{1\over 2}\tr(d\sigma)
		= {1\over 2}\tr \left({M\over\det\Hess\phi} \Hess s\right)
		=  {\Delta_\phi s \over 2\det\Hess\phi}.
	\end{align*}
 The Maxwell paraboloid $\phi_0(x,y)={1\over 2}(x^2+y^2)$ is the canonical 
unit sphere of isotropic geometry, with Hessian $E_2$. Curvatures 
relative to $\phi_0$ are not called ``relative'' and are denoted by the 
symbols $H,K$ instead of $H^\rel,K^\rel$. The observation
	\begin{align*}
	\Delta_\phi \phi
	= \tr(M \Hess \phi)
	= \tr(\wh{\Hess\phi}\Hess\phi)
	= 2\det\Hess\phi
	\end{align*}
 together with the formulas above implies
	\begin{align*}
		K_s  = \det \Hess s, 
	\
		K_\phi = \det \Hess \phi
		%H_s = {\Delta s \over 2},
		%K_s^\rel = {K_s\over K_\phi}
	\implies
		H_s^\rel =  {\Delta_\phi s \over 2 K_\phi}
			= {\Delta_\phi s\over \Delta_\phi \phi}.
	\end{align*}

\paragraph{Relation to Self-supporting Surfaces.}

Summarizing the formulas above, 
we rewrite the balance condition \eqref{eq:conds} as
	\begin{equation}
	2 K_\phi H_s^\rel  = \Delta_\phi s = F.
	\label{equigeo}
	\end{equation}
 Let us draw some conclusions:

\begin{itemize}\itemsep-\parsep

\item Since $H^\rel_\phi=1$ we see that the load $F_\phi=2K_\phi$ is 
admissible for the stress surface $\phi(x,y)$, which is hereby shown as 
self\dash supporting. The quotient of loads yields
	$
	 H_s^\rel = F/F_\phi.
	$

\item If the stress surface coincides with the Maxwell paraboloid, then 
{\em constant loads characterize constant mean curvature surfaces}, 
because we get $K_\phi=1$ and $H_s=F/2$.

\item If $s_1,s_2$ have the same stress potential $\phi$, then 
$H^\rel_{s_1-s_2}=H^\rel_{s_1}-H^\rel_{s_2}=0$, so $s_1-s_2$ is a 
(relative) minimal surface.

\end{itemize}



\subsection{Meshes in Isotropic Geometry} \label{sec:discrete}

A general theory of curvatures of polyhedral surfaces with respect to a 
polyhedral unit sphere was proposed by 
\cite{Pottmann2007b,bobenko-2010-ct}, and its dual complement in isotropic 
geometry was elaborated on in \cite{Pottmann2007}. As illustrated by 
Figure~\ref{fig:christoffel}, the mean curvature of a self\dash supporting 
surface $\SS$ {\em relative} to its discrete Airy stress potential is 
associated with the vertices of $\SS$. It is computed from areas and mixed 
areas of faces in the polar polyhedra $\SS^*$ and $\Phi^*$:
	\begin{align*}
	H^\rel(\vw_i)
	&= {A_i(\SS,\Phi) \over A_i(\Phi,\Phi)},
	\quad\text{where}
	\\
		A_i (\SS,\Phi)
	&=
		\frac{1}{4}
		\sum_{k:f_k\in \text{1-ring}(\vw_i)}
		\det(\vw'^*_k, \ww'^*_{k+1})
		+ \det(\ww'^*_k, \vw'^*_{k+1}).
	\end{align*}
 The prime denotes the projection into the $xy$ plane, and summation is
over those dual vertices which are adjacent to $\vw_i$.
Replacing $\vw_k^*$ by $\ww_k^*$ yields
	$
		A_i (\Phi,\Phi)
	=
	\frac{1}{2}
		\sum
		\det(\ww'^*_k, \ww'^*_{k+1}).
	$

\begin{figure}[h]
 \centering
 \begin{overpic}[width=.8\columnwidth]{fig/christoffel}
	\put(17,12){$\vw_i$}
	\put(0,8){$\SS$}
	\put(0,30){$\Phi$}
	\color{blau}
	\lput(52,18){$\ww_0^*$}
	\lput(65,22){$\vw_0^*$}
	\lput(60,0){$\ww_1^*$}
	\lput(58,32){$\vw_1^*$}
	\put(91,8){$\ww_2^*$}
	\put(82,37){$\vw_2^*$}
	\put(82,21){$\ww_3^*$}
	\put(89,26){$\vw_3^*$}
	\color{gelb}
	\cput(8,37){$f_0^\Phi$}
	\cput(8,18){$f_0^\SS$}
	\cput(13,27){$f_1^\Phi$}
	\cput(13,9){$f_1^\SS$}
	\cput(29,32){$f_2^\Phi$}
	\cput(29,12){$f_2^\SS$}
	\cput(22,42){$f_3^\Phi$}
	\cput(22,22){$f_3^\SS$}
	\cput(70,37){$\SS^*$}
	\cput(80,0){$\Phi^*$}
 \end{overpic}
 \caption{Mean curvature of a vertex $\vw_i$ of $\SS$: Corresponding edges 
of the polar duals $\SS^*$, $\Phi^*$ are parallel, and mean curvature 
according to \protect\cite{Pottmann2007b} is computed from the vertices 
polar to faces adjacent to $\vw_i$. For valence 4 vertices the case of 
zero mean curvature shown here is characterized by parallelity of non\dash 
corresponding diagonals of corresponding quads in $\SS^*,\Phi^*$.}
 \label{fig:christoffel}
 \end{figure}



 \begin{figure*}[t]
	\centering
	%\includegraphics[width=0.24\textwidth]{fig/lilium.png} 
	%\includegraphics[width=0.24\textwidth]{fig/lilium-n.png}
	%\hfill
	%\includegraphics[width=0.24\textwidth]{fig/lilium-pillar-n.png} 
	%\includegraphics[width=0.24\textwidth]{fig/lilium-pq-n.png}
 \centerline{\begin{overpic}[width=.25\textwidth]{fig/lilium0.jpg}
		\cput(40.05,50){\contour{white}{$\downarrow$}}
		\cput(40,56){\contour{white}{$|$}}
		\cput(40,62){\contour{white}{$|$}}
		\cput(40.05,50){$\downarrow$}
		\cput(40,56){$|$}
		\cput(40,62){$|$}
		\cput(40,72){\small impossible feature}
		\put(0,0){(a)}
	\end{overpic}\hfill
	\begin{overpic}[width=.25\textwidth]{fig/lilium-n.jpg}
		\put(0,0){(b)}
		\color{gelb}
		\put(20,64){$\SS$}
	\end{overpic}\hfill
	\begin{overpic}[width=.25\textwidth]{fig/lilium-nstress.jpg}
		\put(0,0){(c)}
		\color{blau}
		\put(25,60){$\Phi$}
		\lput(45,0){$\SS'^*=\Phi'^*$}
	\end{overpic}\hfill
	\begin{overpic}[width=0.25\textwidth]{fig/lilium-pillar-n2.jpg}
		\put(0,0){(d) \small (view from below)}
	\end{overpic}}
 \caption{The top of the Lilium Tower (a) cannot stand as a masonry 
structure, because its central part is concave. Our algorithm finds a 
nearby self-supporting mesh (b) without this impossible feature. (c) shows 
the corresponding Airy mesh $\Phi$ and reciprocal force diagram $\SS'^*$. 
(d) The user can edit the original surface, such as by specifying that the 
center of the surface is supported by a vertical pillar, and the 
self-supporting network adjusts accordingly.}
 \label{fig:Lilium}
 \end{figure*}



\proclaim Proposition.
 If $\Phi$ is the Airy surface of a thrust network $\SS$, then the mean 
curvature of $\SS$ relative to $\Phi$ is computable as
	\begin{equation}
	\label{eq:Hrel}
		H^\rel(\vw_i)
	=
		{\sum_{j\sim i} w_{ij} (s_j-s_i)
		\over \sum_{j\sim i} w_{ij} (\phi_j-\phi_i) }
	=
		{\Delta_\phi s\over \Delta_\phi \phi}\Big|_{\vw_i}.
	\end{equation}

\begin{proof} It is sufficient to show 
	$
	2A_i(\SS,\Phi)
	= \sum_{j\sim i} w_{ij} (s_j-s_i).
	$

 For that, consider edges $\ew'_1,\dots,\ew'_n$ emanating from $\vw_i'$. 
The dual cycles in $\Phi^{*\prime}$ and $\SS^{*\prime}$ without loss of 
generality are given by vertices $(\vw^{*\prime}_1,\dots,\vw^{*\prime}_n)$ 
and $(\ww^{*\prime}_1,\dots,\ww^{*\prime}_n)$, respectively. The latter 
has edges $\ww'^*_{j+1}-\ww'^*_j = w_{ij} J\ew'_j$ (indices modulo $n$).

Without loss of generality $\vw_i=0$, so the vertex $\vw'^*_j$ by 
construction equals the gradient of the linear function $\xw\mapsto 
\<\vw'^*_j,\xw\>$ defined by the properties $\ew'_{j-1}\mapsto 
s_{j-1}-s_i$, $\ew'_j\mapsto s_j-s_i$. Corresponding edge vectors 
$\vw'^*_{j+1}-\vw'^*_j$ and $\ww'^*_{j+1}-\ww'^*_j$ are parallel, because 
$\<\vw'_{j+1}-\vw'_j,\ew'_j\>=(s_j-s_i)-(s_j-s_i)=0$. Expand 
$2A_i(\SS,\Phi)$:
	\begin{align*}
	& \mathrel{\hphantom{=}} \textstyle
		{1\over 2}\sum
		\det(\ww'^*_j, \vw'_{j+1}) + \det(\vw'_j, \ww'^*_{j+1})
	\\
	&=\textstyle
		{1\over 2}\sum
		\det(\ww'^*_j-\ww'^*_{j+1}, \vw'_{j+1})
		+ \det(\vw'_j, \ww'^*_{j+1}-\ww'^*_j)
		\\
	&=\textstyle
		{1\over 2}\sum
		\det( - w_{ij} J\ew'_j, \vw'_{j+1})
	 	+ \det(\vw'_j,w_{ij} J\ew'_j)
	\\
	&= \textstyle
		\sum \det( \vw'_j, w_{ij} J\ew'_j)
	=	 \sum	w_{ij} \< \vw'_j, \ew'_j\>
	= 	 \sum  w_{ij} (s_j-s_i).
	\end{align*}
 Here we have used $\det(\aw,J\bw)=\<\aw,\bw\>$.
 \end{proof}


In order to discretize \eqref{equigeo}, we also need a discrete Gaussian
curvature, usually defined as a quotient of areas which
correspond under the Gauss mapping. We define
	\begin{align*}
	K_\Phi(\vw_i) = {A_i(\Phi,\Phi) \over A_i},
	\end{align*}
 where $A_i$ is the Voronoi area of vertex $\vw_i'$ in the projected mesh
$\SS'$ used in \eqref{eq:deqz}.

{\em Remark:} If the faces of the thrust network $\SS$ are not planar,
the simple trick of introducing additional edges with zero forces in them
makes them planar, and the theory is applicable. In the interest of space, we refrain from elaborating further.


\paragraph{Discrete Balance Equation.}

The discrete version of the balance equation \eqref{equigeo} reads as 
follows:

\proclaim Theorem.
 A simply-connected mesh $\SS$ with vertices $\vw_i=(x_i,y_i,s_i)$
can be put into static equilibrium with vertical nodal forces $A_iF_i$ if
and only if there exists a combinatorially equivalent mesh $\Phi$ with
planar faces and vertices $(x_i,y_i,\phi_i)$, such that curvatures of
$\SS$ relative to $\Phi$ obey
	\begin{equation}
	2 K_\Phi(\vw_i) H^\rel(\vw_i) = F_i
	\label{eq:deqiso}
	\end{equation}
 at every interior vertex and every free boundary vertex $\vw_i$. $\SS$
can be put into compressive static equilibrium if and only if there exists
a convex such $\Phi$.

\begin{proof} The relation between equilibrium forces $w_{ij}\ew_{ij}$ in 
$\SS$ and the polyhedral stress potential $\Phi$ has been discussed above, 
and so has the equivalence ``$w_{ij}\ge 0$ $\iff$ $\Phi$ convex'' (see 
e.g.\ \cite{Ash1988} for a survey of this and related results). It remains 
to show that Equations \eqref{eq:deqtop} and \eqref{eq:deqiso} are 
equivalent. This is the case because the proposition above implies
	$
	2 K(\vw_i) H^\rel(\vw_i) =
	2 \frac{A_i(\Phi,\Phi)}{A_i}
	\frac{A_i(\Phi,\SS)}{A_i(\Phi,\Phi)} =
	{1\over A_i}
	(\sum_{j\sim i} w_{ij} (s_j-s_i))
	= {1\over A_i} A_i F_i.
	$
	\end{proof}

\paragraph{Existence of Discretizations.}

When considering discrete thrust networks as discretizations of continuous 
self\dash supporting surfaces, the following question is important: For a 
given smooth surface $s(x,y)$ with Airy stress function $\phi$, does there 
exist a polyhedral surface $\SS$ in equilibrium approximating $s(x,y)$, 
whose top view is a given planar mesh $\SS'$? We restrict our attention to 
triangle meshes, where planarity of the faces of the discrete stress 
surface $\Phi$ is not an issue. This question has several equivalent 
reformulations:

\begin{itemize}\itemsep-\parsep

\item Does $\SS'$ have a reciprocal diagram whose corresponding Airy 
polyhedron $\Phi$ approximates the continuous Airy potential $\phi$? (if 
the surfaces involved are not simply connected, these objects are defined 
locally).

\item Does $\SS'$ possess a ``perfect'' discrete Laplace\dash Beltrami 
operator $\Delta_\phi$ in the sense of Wardetzky et 
al.~\shortcite{wardetzky07} whose weights are the edge length scalars of 
such a reciprocal diagram?

\end{itemize}

From \cite{wardetzky07} we know that perfect Laplacians exist only on 
regular triangulations which are projections of convex polyhedra. On the 
other hand, previous sections show how to appropriately re\dash 
triangulate: Let $\Phi$ be a triangle mesh convex hull of the vertices 
$(x_i,y_i,\phi(x_i,y_i))$, where $(x_i,y_i)$ are vertices of $\SS'$. Then 
its polar dual $\Phi^*$ projects onto a reciprocal diagram with positive 
edge weights, so $\Delta_\phi$ has positive weights, and the vertices 
$(x_i,y_i,s_i)$ of $\SS$ can be found by solving the discrete Poisson 
problem $(\Delta_\phi s)_i=A_iF_i$.

Assuming the discrete $\Delta_\phi$ approximates its continuous 
counterpart, this yields a mesh approximating $s(x,y)$, and we conclude:
{\it A smooth self-supporting surface can be approximated by a discrete
self\dash supporting triangular mesh for any sampling of the surface.}


\begin{figure*}[t]
	%\includegraphics[width=0.24\textwidth]{fig/build.png}
	%\includegraphics[width=0.24\textwidth]{fig/build-n.png}
	%\hfill
	%\includegraphics[width=0.24\textwidth]{fig/build-edited.png}
	%\includegraphics[width=0.24\textwidth]{fig/build-edited-n.png}
\centerline{
	\begin{overpic}[width=0.25\textwidth]{fig/build-white.jpg}
	\put(0,0){(a)}
	\end{overpic}\hfill
	\begin{overpic}[width=0.25\textwidth]{fig/build-n.jpg}
	\put(0,0){(b)}
	\end{overpic}\hfill
	\begin{overpic}[width=0.25\textwidth]{fig/build-edited-white.jpg}
	\put(0,0){(c)}
	\end{overpic}\hfill
	\begin{overpic}[width=0.25\textwidth]{fig/build-edited-n.jpg}
	\put(0,0){(d)}
	\end{overpic}}
	\vskip-1ex

\caption{The user-designed reference mesh (a) is not self-supporting, 
but our algorithm finds a nearby perturbation of the reference surface 
(b) that is in equilibrium. As the user makes edits to the 
reference surface (c), the thrust network automatically adjusts 
(d). \label{fig:vault}}

	\end{figure*}

\begin{table*}[t]
	\medskip
\begin{tabular}{@{}llccccc@{}}
\hline
	\textit{Example} 
		& \textit{Figure} 
		& \textit{Vertices}  
		& \textit{Edges} 
			& \textit{Time} (s) 
			& \textit{Iterations} 
			& \textit{Max Rel Error} \\
	\hline
		Top of Lilium Tower 
		& Fig. \ref{fig:Lilium}b
		& 1201 
		& 3504 
			& 21.6 
			& 9 
			& $4.2 \times 10^{-5}$
	\\ Top of Lilium Tower (with pillar) 
		& Fig. \ref{fig:Lilium}d
		& 1200 
		& 3500 
			& 26.5 
			& 10 
			& $8.5 \times 10^{-5}$
	\\ Freeform Structure with Two Pillars 
		& Fig. \ref{fig:cas} 
		& 1535 
		& 2976 
			& 17.0 
			& 21 
			& $2.7\times 10^{-5}$
	\\ Swiss Cheese 
		& Fig. \ref{fig:cheese2} 
		& 2358 
		& 4302 
			& 19.5 
			& 9 
			& $3.0 \times 10^{-4}$
	\\ Brick Domes 
		& Fig. \ref{fig:removal} 
		& 752 
		& 2165 
			& 8.0 
			& 9 
			& $5.8 \times 10^{-5}$
	\\ Structural Glass 
		& Fig. \ref{fig:structural} 
		& 527 
		& 998 
			& 5.7 
			& 25 
			& $2.4\times 10^{-5}$
	\\ \hline
  \end{tabular}
	\medskip
	\caption{Numerical details about the examples throughout this 
paper. We show the -clock time needed by an Intel Xeon 2.3GHz 
desktop PC with 4 GB of RAM to find a self\dash supporting thrust network and 
associated stress surface from the example's reference mesh; we also give 
the number of outer iterations of the four steps in \secref{sec:opt}. The 
maximum relative error is the dimensionless relative error in force 
equilibrium defined by $\max_i \| A_i F_i - \sum\nolimits_{j\sim i} w_{ij} 
(\vw_j-\vw_i) \|/\|A_i F_i\|$, where the maximum is taken over interior vertices $\vw_i$. 
} \label{table:data}


\end{table*}


\section{Thrust Networks from Reference Meshes} \label{sec:opt} 

Consider now the problem of taking a given reference mesh, say $\RR$, and 
finding a combinatorially equivalent mesh $\SS$ in static equilibrium 
approximating $\RR$. The loads on $\SS$ include user-prescribed loads as 
well as the dead load caused by the mesh's own weight. Conceptually, 
finding $\SS$ amounts to minimizing some formulation of distance between 
$\RR$ and $\SS$, subject to constraints \eqref{eq:deqtop}, 
\eqref{eq:deqz}, and $w_{ij} \geq 0$. For any choice of distance this 
minimization will be a nonlinear, non-convex, inequality-constrained 
variational problem. Our experience with black-box solvers is that 
they perform very well for surfaces without complex geometry or for
polishing reference meshes close to self-supporting, but fail to converge
in reasonable time for more complicated shapes such as the swiss cheese example (Fig. \ref{fig:cheese}).
We therefore propose the following specialized, staggered linearization for solving the optimization problem:

\begin{enumerate}\itemsep-\parsep\setcounter{enumi}{-1}

\item Start with an initial guess $\SS = \RR$.

\item \label{step2} Estimate the self\dash load on the vertices of $\SS$, 
using their current positions.

\item \label{step3} Fixing $\SS$, locally fit an associated stress surface $\Phi$.

\item \label{step4} Alter positions $\vw_i$ to improve the fit.

\item Repeat from Step~\ref{step2} until convergence.

\end{enumerate}

This staggered approach shares the several advantages of solving the full nonlinear problem: a nearby self\dash supporting surface
is found given only a suggested reference shape, without needing to single one of
the many possible top view reciprocal diagrams or needing to specify boundary
tractions -- these are found automatically during optimization. Although providing
an initial top view graph with good combinatorics remains important, by not fixing the top view our approach
allows the thrust network to slide both vertically and tangentially to the ground,
essential to finding faithful thrust networks for surfaces with free boundary conditions.

\paragraph{Step~\ref{step2}: Estimating Self-Load.}

The dead load due to the surface's own weight depends not only on the top 
view of $\SS$, but also on the surface area of its faces. To avoid adding 
nonlinearity to the algorithm, we estimate the load coefficients $F_i$ at 
the beginning of each iteration, and assume they remain constant until the 
next iteration. We estimate the load $A_iF_i$ associated with each 
vertex by calculating its Voronoi surface area on each of its incident faces 
(note that this surface area is distinct from $A_i$, the vertex's Voronoi 
area on the top view), and then multiplying by a user-specified surface density $\rho$.

\paragraph{Step~\ref{step3}: Fit a Stress Surface.}

In this step, we fix $\SS$ and try to fit a stress surface $\Phi$ 
subordinate to the top view $\SS'$ of the primal mesh. We do so by 
searching for dihedral angles between the faces of $\Phi$ which minimize, 
in the least-squares sense, the error in force equilibrium 
\eqref{eq:deqiso} and local integrability of $\Phi$. Doing so is 
equivalent to minimizing the squared residuals of Equations 
\eqref{eq:deqz} and \eqref{eq:deqtop}, respectively, with the positions 
held fixed. We define the {\em equilibrium energy}
	\begin{align}
	E = \sum\nolimits_i \Big\| \Forcevector -
		\sum\nolimits_{j\sim i} w_{ij} (\vw_j-\vw_i) \Big\|^2,
	\label{eq:eenergy}
	\end{align}
 where the outer sum is over the interior and free boundary vertices,
and we solve
	\begin{align}
	\min_{w_{ij}} E,
	\quad
	\textrm{s.t.}\ \
		0 \leq w_{ij} \leq w_{\max}.
	\label{eq:wbounds}
	\end{align}
 Here $w_{\max}$ is an optional maximum weight we are willing to assign 
(to limit the amount of stress in the surface). This convex, sparse, 
box-constrained least-squares problem \cite{BCLS} always has a solution. 
If the objective is $0$ at this solution, the faces of $\Phi$ locally 
integrate to a stress surface satisfying \eqref{eq:deqiso}, and this $\Phi$ 
certifies that $\SS$ is self-supporting -- we are done. Otherwise, $\SS$ 
is not self-supporting and its vertices must be moved.

\paragraph{Step~\ref{step4}: Alter Positions.} In the previous step we fit 
as best as possible a stress surface $\Phi$ to $\SS$. There are two 
possible kinds of error with this fit: the faces around a vertex 
(equivalently, the reciprocal diagram) might not close up; and the 
resulting stress forces might not be exactly in equilibrium with the 
loads. These errors can be decreased by modifying the top view and heights 
of $\SS$, respectively. It is possible to simply solve for new vertex 
positions that put $\SS$ in static equilibrium, since Equations 
\eqref{eq:deqtop} and \eqref{eq:deqz} with $w_{ij}$ fixed form a square 
linear system that is typically nonsingular.

While this approach would yield a self-supporting $\SS$, this mesh is 
often far from the reference mesh $\RR$, since any local errors in the 
stress surface from Step~\ref{step3} amplify into global errors in $\SS$. 
We propose instead to look for new positions that decrease the imbalance 
in the stresses and loads, while also penalizing drift away from the 
reference mesh:
	\begin{align*}
	\min_{\vw} E
	+ \alpha \sum\nolimits_i
		\<\nw_i, \vw_i - \vw^0_i \>^2
		+ \beta \big\|\vw - \vw^0_P\big\|^2,
	\end{align*}
 where $\vw^0_i$ is the position of the $i$-th vertex at the start of this 
step of the optimization, $\nw_i$ is the starting vertex normal (computed 
as the average of the incident face normals), $\vw^0_P$ is the projection 
of $\vw^0$ onto the reference mesh, and $\alpha > \beta$ are penalty 
coefficients that are decreased every iteration of Steps 
\ref{step2}--\ref{step4} of the algorithm. The second term allows $\SS$ to 
slide over itself (if doing so improves equilibrium) but penalizes drift 
in the normal direction. The third term, weaker than the second, 
regularizes the optimization by preventing large drift away from the 
reference surface or excessive tangential sliding.

\paragraph{Implementation Details.}

Solving the weighted least\dash squares problem of Step~\ref{step4} 
amounts to solving a sparse, symmetric linear system. While the MINRES 
algorithm~\cite{paige75} is likely the most robust algorithm for solving 
this system, in practice we have observed that the method of conjugate 
gradients works well despite the potential ill-conditioning of the 
objective matrix. 

\begin{figure*}[t]
	\centerline{\includegraphics[height=0.14\textwidth]{fig/cas.jpg}\relax
	\includegraphics[height=0.14\textwidth]{fig/cas-n.jpg}\hfill
	\begin{minipage}[b]{.7\columnwidth} \caption{A freeform surface 
(left) needs adjustments around the entrance arch and between the two 
pillars in order to be self-supporting; our algorithm finds the nearby 
surface in equilibrium (right) that incorporates these 
changes.}\label{fig:cas}
	\end{minipage}}
	 \end{figure*}

  \begin{figure*}[t]
  \newdimen\tmplen\tmplen=0.10\textwidth
  \centerline{\includegraphics[width=2.57\tmplen]{fig/remove-1n.jpg}\hfill
	  \includegraphics[width=2.57\tmplen]{fig/remove-2n.jpg}\hfill
	  \raise.43\tmplen
	\hbox{\includegraphics[width=2.23\tmplen]{fig/remove-3n.jpg}}\hfill
	  \raise.43\tmplen
	\hbox{\includegraphics[width=2.24\tmplen]{fig/remove-4n.jpg}}}
   \vskip-6ex
   \leftline{(a)\hskip 2.5\tmplen (b)\hskip 2.5\tmplen (c)\hskip 2.2\tmplen
		(d)}
   \vskip1.5ex

\caption{Destruction sequence. We simulate removing small parts of masonry 
(their location is shown by a yellow ball) and 
the falling off of further pieces which are no longer supported after 
removal. For this example, removing a certain small number 
of single bricks does not affect stability (a,b). Removal of material at a 
certain point (yellow ball in (b))
will cause a greater part of the structure to collapse,
as seen in (c). (d) shows the result after one more removal (all images show 
the respective thrust networks, not the reference surface).} 
\label{fig:removal}
\end{figure*}

\paragraph{Limitations.}

This algorithm is not guaranteed to always converge; this fact is not 
surprising from the physics of the problem (if the boundary of the 
reference mesh encloses too large of a region, $w_{\max}$ is set too low, 
and the density of the surface too high, a thrust network in equilibrium 
simply does not exist -- the vault is too ambitious and cannot be built to 
stand; pillars are needed.)

We can, however, make a few remarks. Step~\ref{step3} always decreases the 
equilibrium
 energy $E$ of Equation~\eqref{eq:eenergy}
	\nix{$$E=\sum_i \Big\| \Forcevector
		- \sum_{j\sim i} w_{ij} (\vw_j - \vw_i)\Big\|^2$$}
 and Step~\ref{step4} does as well as $\beta \to 0$. Moreover, as $\alpha 
\to 0$ and $\beta \to 0$, Step~\ref{step4} approaches a linear system with 
as many equations as unknowns; if this system has full rank, its solution 
sets $E=0$. These facts suggest that the algorithm should generally 
converge to a thrust network in equilibrium, provided that 
Step~\ref{step2} does not increase the loads by too much at every 
iteration, and this is indeed what we observe in practice. One case where 
this assumption is guaranteed to hold is if the thickness of the surface 
is allowed to freely vary, so that it can be chosen so that the surface 
has uniform density over the top view.

If the linear system in Step~\ref{step4} is singular and infeasible, the 
algorithm can stall at $E > 0$. This failure occurs, for instance, when an 
interior vertex has height $z_i$ lower than all of its neighbors, and 
Step~\ref{step3} assigns all incident edges to that vertex a weight of 
zero: clearly no amount of moving the vertex or its neighbors can bring 
the vertex into equilibrium. We avoid such degenerate configurations by 
bounding weights slightly away from zero in \eqref{eq:wbounds}, trading 
increased robustness for slight smoothing of the resulting surface. Attempting
to optimize meshes that have self-intersecting top views (i.e., aren't height fields),
have too many impossible features, or are insufficiently supported by fixed
boundary points can also result in errors and instability.





\section{Results}
\label{sec:design}

\paragraph{Interactive Design of Self-Supporting Surfaces.}

The optimization algorithm described in the previous section forms the 
basis of an interactive design tool for self-supporting surfaces. Users 
manipulate a mesh representing a reference surface, and the computer 
searches for a nearby thrust network in equilibrium (see e.g.\ 
Figure~\ref{fig:vault}). Features of the design tool include:

\begin{itemize}\itemsep-\parsep

\item Handle-based 3D editing of the reference mesh using Laplacian 
coordinates~\cite{Lipman2004,Sorkine2003} to extrude vaults, insert 
pillars, and apply other deformations to the reference mesh. Handle-based 
adjustments of the heights, keeping the top view fixed, and deformation of 
the top view, keeping the heights fixed, are also supported. The thrust 
network adjusts interactively to fit the deformed positions, giving the 
usual visual feedback about the effects of edits on whether or not the 
surface can stand.

\item Specification of boundary conditions. Points of contact between the 
reference surface and the ground or environment are specified by 
``pinning'' vertices of the surface, specifying that the thrust network 
must coincide with the reference mesh at this point, and relaxing the 
condition that forces must be in equilibrium there.

\item Interactive adjustment of surface density $\rho$, external loads, 
and maximum permissible stress per edge $w_{\textrm{max}}$, with visual 
feedback of how these parameters affect the fitted thrust network.

\item Upsampling of the thrust network through Catmull\dash Clark 
subdivision \nix{\cite{catmull78}}
and polishing of the resulting refined thrust 
network using optimization \secref{sec:opt}.

\item Visualization of the stress surface dual to the thrust network 
and corresponding reciprocal diagram.

\end{itemize}

 \begin{figure}[b]
  \centerline{\begin{overpic}[width=0.5\columnwidth]{fig/cheese.jpg}
		\put(0,5){(a)}
	\end{overpic}\relax
	\begin{overpic}[width=0.5\columnwidth]{fig/cheese-n.jpg}
		\put(0,5){(b)}
	\end{overpic}}
	\vskip-1.5ex
  \caption{A mesh with holes (a) requires large deformations
to both the top view and heights to render it self\dash supporting
(b)}\label{fig:cheese2}
  \end{figure}

\begin{figure*}[t]
\begin{overpic}[width=0.23\textwidth]{fig/lilium-vf.jpg}
		\put(0,0){(a)}
	\end{overpic}\relax
	\begin{overpic}[width=0.23\textwidth]{fig/lilium-pq.jpg}
		\put(0,0){(b)}
	\end{overpic}\relax
	\begin{overpic}[width=0.23\textwidth]{fig/lilium-pq-n.jpg}
		\put(0,0){(c)}
	\end{overpic}\hfill
	\begin{minipage}[b]{.30\textwidth}
	\caption{Planar quad remeshing of the ``Lilium tower'' surface of 
Figure~\protect\ref{fig:Lilium}. (a) Principal directions which are found
as eigenvectors of $(\Hess\phi)^{-1}\Hess s$. (b) Quad mesh 
guided by principal directions is almost planar and almost self\dash 
supporting. (c) Small changes achieve both properties.}
 \label{fig:lilium:pq}\end{minipage}

\bigskip

	\includegraphics[height=0.15\textwidth]{fig/cas-vf.jpg}\hfill
	\includegraphics[height=0.15\textwidth]{arch-fig/1roo62.jpg}\hfill
	\includegraphics[height=0.18\textwidth]{arch-fig/1roo58.jpg}
	\caption{Planar quad remeshing of the surface of 
Figure~\protect\ref{fig:cas}. Left: Relative principal directions. Center: The 
result of optimization is a self\dash supporting PQ mesh, which guides a 
moment\dash free steel\slash glass construction. Right: Interior view.}
	\label{fig:cas:pq}

  \end{figure*}

\def\sparagraph#1{\par\noindent{\it #1}}

\sparagraph{{\sf\bfseries Examples.} Vault with Pillars:} 
As an example of the design and 
optimization workflow, consider a rectangular vault with six pillars, free 
boundary conditions along one edge, fixed boundary conditions along the 
others, and a tower extruded from the top of the surface (see Figure 
\ref{fig:vault}). This surface is neither convex nor simply connected, and 
exhibits a mix of boundary conditions, none of which cause our algorithm 
any difficulty; it finds a self-supporting thrust network near the 
designed reference mesh. The user is now free to make edits to the 
reference mesh, and the thrust network adapts to these edits, providing 
the user feedback on whether these designs are physically realizable.


\sparagraph{Example: Top of the Lilium Tower.}
Consider the top portion of the steel-glass exterior surface of the Lilium 
Tower, which is currently being built in Warszaw (see Figure 
\ref{fig:Lilium}). This surface contains a concave part with local minimum 
in its interior and so cannot possibly be self-supporting. Given this 
surface as a reference mesh, our algorithm constructs a nearby thrust 
network in equilibrium without the impossible feature. The user can then 
explore how editing the reference mesh -- adding a pillar, for example -- 
affects the thrust network and its deviation from the reference surface.

\sparagraph{Example: Freeform Structure with Two Pillars.}
Suppose an architect's experience and intuition has permitted the design 
of a nearly self\dash supporting freeform surface (Figure \ref{fig:cas}).
Our algorithm reveals those edits needed to make the 
structure sound -- principally around the entrance arch, and the area 
between the two pillars.

\sparagraph{Example: Destruction Sequence.}
In Figure~\ref{fig:removal} we simulate removing parts of masonry and 
the falling off of further pieces which are no longer supported after 
removal.
This is done by deleting the 1-neighborhood of a vertex 
and solving for a new thrust network in compressive equilibrium 
close to the original reference surface. We delete those parts of the 
network which deviate too much and are no longer contained in the masonry 
hull, and iterate. 




\sparagraph{Example: Swiss Cheese.}
Cutting holes in a self-supporting surface interrupts force flow lines and 
causes dramatic global changes to the surface stresses, often to the point 
that the surface is no longer in equilibrium. Whether a given surface with 
many such holes can stand is far from obvious. Figure~\ref{fig:cheese2} 
shows such an implausible and unstable surface; our optimization finds a 
nearby, equally implausible but stable surface without difficulty (see 
Figures~\ref{fig:cheese} and \ref{fig:cheese2}).


\section{Special Self-Supporting Surfaces} \label{sec:special}

\paragraph{PQ Meshes.}

Meshes with {\em planar} faces are of particular interest in architecture, 
so in this section we discuss how to remesh a given thrust network in 
equilibrium such that it becomes a quad mesh with planar faces (again in 
equilibrium). If this mesh is realized as a steel\dash glass construction,
it is self\dash supporting in its beams alone, with no forces exerted on
the glass (this is the usual manner of using glass). The beams constitute
a self\dash supporting structure which is in perfect force equilibrium
(without moments in the nodes) if only the deadload is applied.


	\begin{figure}[h]
  \centerline{\begin{overpic}[width=0.33\columnwidth]{fig/vault.jpg}
	\put(0,0)({(a)}
	\end{overpic}\hfill
  \begin{overpic}[width=0.33\columnwidth]{fig/vault-flat.jpg}
	\put(0,0)({(b)}
	\end{overpic}\hfill
  \begin{overpic}[width=0.33\columnwidth]{fig/vault-pq.jpg}
	\put(0,0)({(c)}
	\end{overpic}}
	\caption{Directly enforcing planarity of the faces of even a very 
simple self-supporting quad-mesh vault (a) results in a surface far 
removed from the original design (b). Starting instead from a remeshing of 
the surface with edges following relative principal curvature directions 
yields a self-supporting, PQ mesh far more faithful to the original (c).}
	\label{fig:badpq}
\end{figure}


Taking an arbitrary non-planar 
quad mesh and attempting naive, simultaneous enforcement of planarity and 
static equilibrium -- either by staggering a planarity optimization step every outer
iteration, or adding a planarity penalty term to the position update -- does not 
yield good results, as shown in Figure~\ref{fig:badpq}. Indeed, as we will see later
in this section, such a planar perturbation of a thrust network is not expected to 
generally exist.

Consider a planar quad mesh $\SS$ with vertices
$\vw_{ij}=(x_{ij},y_{ij},s_{ij})$ which approximates a given continuous
surface $s(x,y)$. It is known that $\SS$ must approximately follow a network
of conjugate curves in the surface (see e.g.\ \cite{Liu2006}). We can derive
this condition in an elementary way as follows: Using a Taylor expansion, we 
compute the volume of the convex hull of the quadrilateral $\vw_{ij}$, 
$\vw_{i+1,j}$, $\vw_{i+1,j+1}$, $\vw_{i,j+1}$, assuming the vertices lie 
exactly on the surface $s(x,y)$. This results in
	\begin{align*}
	&\textstyle
	\text{vol} =
	{1\over 6}\det(\aw_1,\aw_2) \cdot 
		\left((\aw_1)^T\,\Hess s\,\aw_2 \right)+ \cdots,
	\\
	\text{where}\
	& \textstyle
	\aw_1={x_{i+1,j}-x_{ij}\choose y_{i+1,j}-y_{ij}},\quad
	\aw_2={x_{i,j+1}-x_{ij}\choose y_{i,j+1}-y_{ij}},
	\end{align*}
 and the dots indicate higher order terms. We see that planarity requires 
$(\aw_1)^T\,\Hess s\,\aw_2=0$. In addition to the mesh $\SS$ approximating 
the surface $s(x,y)$, the corresponding polyhedral Airy surface $\Phi$ 
must approximate $\phi(x,y)$; thus we get the conditions
	\begin{align*}
	(\aw_1)^T\,\Hess s\,\,\aw_2=
	(\aw_1)^T\,\Hess \phi\,\,\aw_2= 0.
	\end{align*}
 $\aw_1,\aw_2$ are therefore eigenvectors of $(\Hess\phi)^{-1}\Hess s$. In 
view of \S\ref{sec:smooth}, $\aw_1,\aw_2$ indicate the principal 
directions of the surface $s(x,y)$ relative to $\phi(x,y)$.


\nix{\begin{figure}[h]
	\begin{overpic}[width=.45\columnwidth]{fig/curvature1}
		\put(55,5){$\phi(x,y)$}
		\put(55,51){$s(x,y)$}
	\end{overpic}\hfill
	\begin{minipage}[b]{.55\columnwidth}
 \caption{Planar quad remeshing of a self-supporting surface $s(x,y)$ with 
stress potential $\phi$ is guided by the principal curvature directions of 
$s$ relative to $\phi$ (found from eigenvectors of 
$(\protect\Hess\phi)^{-1}\protect\Hess s$).}
 \label{fig:relative}
	\end{minipage}
 \end{figure}}




In the discrete case, where $s,\phi$ are not given as continuous surfaces, 
but are represented by a mesh in equilibrium and its Airy mesh, we use the 
techniques of Schiftner~\shortcite{Schiftner2007} and Cohen\dash Steiner 
and Morvan \shortcite{Cohen-Steiner2003} to approximate the Hessians 
$\Hess s$, $\Hess\phi$, compute principal directions as eigenvectors of 
$(\Hess\phi)^{-1}\Hess s$, and subsequently find meshes $\SS,\Phi$ 
approximating $s,\phi$ which follow those directions. Global optimization 
can now polish $\SS,\Phi$ to a valid thrust network with discrete stress 
potential, where before it failed: we do so by taking the planarity energy
$\sum_f (2\pi - \theta_f)^2$, where the sum runs over faces and $\theta_f$ is the
sum of the interior angles of face $f$, linearizing it at every iteration, and
adding it to the objective function of the position update (Step \ref{step4}).
Convexity of $\Phi$ ensures that $\SS$ is self\dash supporting. 


Note that for each $\Phi$, the relative principal curvature directions give the 
\emph{unique} curve network along which a planar quad discretization of a 
self\dash supporting surface is possible. Other networks lead to results
like the one shown by Figure~\ref{fig:badpq}.
Figures~\ref{fig:lilium:pq} and \ref{fig:cas:pq} 
further illustrate the result of applying this procedure to 
self-supporting surfaces.




{\em Remark:} When remeshing a given shape by planar quad meshes, we know
that the circular and conical properties require that the mesh follows the
ordinary, Euclidean principal curvature directions \cite{Liu2006}. It is
remarkable that the self\dash supporting property in a similar manner
requires us to follow certain {\em relative} principal directions.
Practictioners' observations regarding the beneficial statics properties
of principal directions can be explained by this analogy, because the
relative principal directions are close to the Euclidean ones, if the
stress distribution is uniform and $\|\nabla s\|$ is small.



\paragraph{Koenigs Meshes.}

Given a self\dash supporting thrust network $\SS$ with stress surface
$\Phi$, we ask the question:
Which vertical perturbation $\SS+\RR$ is self\dash supporting, with the same
loads as $\SS$? As to notation, all involved meshes $\SS,\RR,\Phi$ have the
same top view, and arithmetic operations refer to the respective $z$
coordinates
$s_i,r_i,\phi_i$ of vertices.

The condition of equal loads then is expressed as
$\Delta_\phi(s+r)=\Delta_\phi s$ in terms of Laplacians or
as $H^\rel_\SS=H^\rel_{\SS+\RR}$ in terms of mean curvature, and is equivalent
to 
	\begin{align*}
	\Delta_\phi r = 0, \quad \mbox{i.e.,}\quad
	H^\rel_\RR = 0.
	\end{align*}
 So $\RR$ is a {\em minimal surface} relative to $\Phi$.  While in the
triangle mesh case there are enough degrees of freedom for nontrivial
solutions, the case of planar quad meshes is more intricate:
Polar polyhedra $\RR^*,\Phi^*$ have to be
Christoffel duals of each other \cite{Pottmann2007}, as illustrated by
Figure~\ref{fig:christoffel}. Unfortunately not all quad meshes
have such a dual; the condition is that the mesh is {\em Koenigs}, i.e.,
the derived mesh formed by the intersection points of diagonals of faces
again has planar faces \cite{bobenko-2008-ddg}.

\begin{figure}[h]
	\begin{overpic}[width=\columnwidth]{fig/enneper2.jpg}
		\color{gelb}
		\lput(12,0){$\Phi+\alpha\RR$}
		\lput(45,0){$\Phi$}
		\color{blau}
		\lput(77,0){$\RR$}
	\end{overpic}
 \caption{A ``Koebe'' mesh  $\Phi$ is self\dash supporting for unit dead
load. An entire family of self-supporting meshes with the same top view
is defined by $\SS_\alpha=\Phi+\alpha\RR$, where $\RR$ is chosen as $\Phi$'s 
Christoffel\dash dual.} \label{fig:enneper}
	\end{figure}



\paragraph{Koebe meshes.}

An interesting special case occurs if $\Phi$ is a {\it Koebe mesh} of 
isotropic geometry, i.e., a PQ mesh whose edges touch the Maxwell 
paraboloid. Since $\Phi$ approximates the Maxwell paraboloid, we get 
$2K(\vw_i)H^{\rel}(\vw_i)$ $ \approx 1$ and $\Phi$ consequently is 
self\dash supporting for unit load. Applying the Christoffel dual 
construction described above yields a minimal mesh $\RR$ and a family of 
meshes $\Phi+\alpha\RR$ which are self\dash supporting for unit load (see 
Figure~\ref{fig:enneper}).


\section{Conclusion and Future Work}

\paragraph{Conclusion.}

This paper builds on relations between statics and geometry, some of which 
have been known for a long time, and connects them with newer methods of 
discrete differential geometry, such as discrete Laplace operators and 
curvatures of polyhedral surfaces. We were able to find efficient ways of 
modeling self\dash supporting freeform shapes, and provide architects and 
engineers with an interactive tool for evaluating the 
statics of freeform geometries. The self\dash supporting property of a 
shape is directly relevant for freeform masonry. The actual thrust 
networks we use for computation are relevant e.g.\ for steel 
constructions, where equilibrium of deadload forces implies absence of 
moments. This theory and accompanying algorithms thus constitute a new 
contribution to architectural geometry, connecting statics and geometric 
design.

\paragraph{Future Work.}

There are several directions of future research. One is to incorporate 
non\dash manifold meshes, which occur naturally when e.g.\ supporting 
walls are introduced. It is also obvious that non\dash vertical loads, 
e.g.\ wind load, play a role. There are also some directions to pursue in 
improving the algorithms, for instance adaptive remeshing in problem 
areas. Probably the interesting connections between statics properties and 
geometry are not yet exhausted, and we would like to propose the {\em 
geometrization} of problems as a strategy for their solution.

\paragraph*{Acknowledgements.}

This work was very much inspired by Philippe Block's plenary lecture
at the 2011 Symposium on Geometry Processing in Lausanne. Several 
illustrations (the maximum load example of Figure
\ref{fig:load} and the destruction sequence of Figure \ref{fig:removal})
have real\dash world analogues on his web page \cite{catalan}.

\begin{figure*}[t]
	\begin{overpic}[width=.49\columnwidth]{fig/dome-unloaded.jpg}
		\small
		\cput(30,0){14\,m}
		\put(0,-4){\hbox to .3\columnwidth{\leftarrowfill}}
		\lput(100,-4){\hbox to .3\columnwidth{\rightarrowfill}}
		\put(0,47){shell thickness 0.1\,m, $\rho=2,500$\,kg/m$^3$}
	\end{overpic}\hfill
	\begin{overpic}[width=.49\columnwidth]{fig/dome-loaded.jpg}
		\small
		\put(60,44){11,000\,kg}
	\end{overpic}\hfill
	\begin{minipage}[b]{.5\textwidth}
	\caption{Stability Test. Left: Coloring and cross\dash section
of edges visualize the magnitude of forces in a thrust network which is in
equlibrium with this dome's dead load.
Right: When an additional load is applied, there exists a corresponding
compressive thrust network which is still contained in the masonry hull
of the original dome. This implies stability of the dome under that load.}
\label{fig:load}
\end{minipage}

	\includegraphics[width=.23\textwidth]{fig/dome2-00.jpg}\hfill
	\includegraphics[width=.25\textwidth]{fig/dome2-05.jpg}\hfill
	\includegraphics[width=.25\textwidth]{fig/dome2-10.jpg}\hfill
	\includegraphics[width=.25\textwidth]{fig/dome2-20.jpg}
	\caption{Stability test similar to Figure~\protect\ref{fig:load},
but with a shell thickness of 1\,m, in order to better visualize the way
the thrust network starts to leave the masonry hull as the load increases.
Additional loads are 0\,kg, 5,000\,kg, 10,000\,kg, and 20,000\,kg, resp., from
left to right.}
\end{figure*}

\bibliographystyle{acmsiggraph}

	%\let\otb=\thebibliography
	%\def\thebibliography#1{\otb{#1}\itemsep-5pt\footnotesize}

\bibliography{selfsupporting}

\nix{\begin{figure}[h]
	\begin{overpic}[width=.49\columnwidth]{fig/dome-unloaded.jpg}
		\small
		\cput(50,4){14\,m}
		\put(0,0){\hbox to .3\columnwidth{\leftarrowfill}}
		\lput(100,0){\hbox to .3\columnwidth{\rightarrowfill}}
		\put(0,60){shell thickness 0.1\,m}
		\put(0,67){specific weight 2,500\,kg/m$^3$}
	\end{overpic}\hfill
	\begin{overpic}[width=.49\columnwidth]{fig/dome-loaded.jpg}
		\small
		\put(60,60){11,000\,kg}
	\end{overpic}
	\caption{Stability Test. (a) Coloring and cross\dash section
of edges visualize the forces in a thrust network which is in equlibrium
with this dome's dead load.
(b) When an additional load is applied, there exists a corresponding
compressive thrust network which is still contained in the masonry hull
of th e original dome. This implies stability
of the dome under that load.}
\label{fig:load}
\end{figure}}

\begin{figure}[h]
	\vspace*{2cm}
	%\includegraphics[width=\columnwidth]{arch-fig/schneider84.jpg}\\[-1em]
	\includegraphics[width=\columnwidth]{arch-fig/schneider87.jpg} 
	%\includegraphics[width=\columnwidth]{arch-fig/structural79.jpg} 
	% \centerline{\includegraphics[width=.65\textwidth]
	%	{arch-fig/schneiderbeide.jpg} 
	%\hfill
	%\begin{minipage}[b]{.33\textwidth}
 \caption{Glass as a structural element can support stresses up to, say, 
30\,MPa. We propose steel\slash glass constructions which utilize the 
structural properties of glass by first solving for a self\dash supporting 
thrust network such that forces do not exceed the maximum values, and 
subsequent remeshing of this surface by a planar quad mesh (not 
necessarily self\dash supporting itself). Since this surface is very 
close to a self\dash supporting shape, joints will experience low bending 
and torsion moments.\vspace*{2em}} \label{fig:structural}
	%\end{minipage}}\vspace*{-2em}
\end{figure}


\end{document}

% Work into limitations?

\subsubsection{Possible pitfall}
Though the above optimization problem always has a solution, there is one unpleasant situation can arise: the optimum weights around \emph{all} edges adjacent to a vertex might be 0. I've observed this occur when the reference mesh has an interior local minimum at vertex $i$: obviously such a reference mesh cannot be a thrust network in equilibrium. If the neighbors of $i$ have few edges (such as in a quad mesh) the optimal weights for the edges around $i$ will be positive, since these edges are "needed" by $i$'s neighbors to put them in equilibrium. On the other hand, if $i$'s neighbors are edge-rich (such as in a triangle mesh) occasionally the best solution is to set all of $i$'s edges to have weight 0, effectively deleting vertex $i$ from the thrust network. A vertex whose surrounding edge weights are all zero (or near-zero) clearly can never satisfy \eqref{eq:dcond}, and giving the next step of the algorithm such a vertex as input causes numerical instabilities. Therefore in the current code I simply delete such vertices from the thrust network.

Reference meshes with large flat regions similarly cause numerical problems: again, in such regions no amount of altering the weights can improve the error in equation \eqref{eq:dcond}, leading the above optimization to sometimes assign many near-zero weights in such regions.


